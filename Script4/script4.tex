\documentclass[../main.tex]{subfiles}

\pagestyle{main}
\renewcommand{\chaptermark}[1]{\markboth{\chaptername\ \thechapter}{#1}}
\setcounter{chapter}{3}

\begin{document}




\chapter{The Topology of a Continuum}\label{sct:4}
\section{Journal}
\begin{definition}\label{dfn:4.1}\marginnote{11/3:}
    A subset of a continuum is \textbf{closed} if it contains all of its limit points.
\end{definition}

\begin{theorem}\label{trm:4.2}
    The sets $\emptyset$ and $C$ are closed.
    \begin{proof}
        We will address the two sets individually.\par
        To prove that $\emptyset$ is closed, Definition \ref{dfn:4.1} tells us that it will suffice to show that $\emptyset\subset C$ and $\emptyset$ contains all of its limit points. By Exercise \ref{exr:1.10}, $\emptyset\subset C$. We now prove that $\emptyset$ has no limit points. Suppose for the sake of contradiction that some point $p\in C$ is a limit point of $\emptyset$. Then by Definition \ref{dfn:3.13}, for all regions $R$ with $p\in R$, $R\cap(\emptyset\setminus\{p\})\neq\emptyset$. But clearly, $R\cap(\emptyset\setminus\{p\})=R\cap\emptyset=\emptyset$, a contradiction. Therefore, since $\emptyset$ has no limit points, the statement "$\emptyset$ contains all of its limit points" is vacuously true.\par
        To prove that $C$ is closed, Definition \ref{dfn:4.1} tells us that it will suffice to show that $C\subset C$ and $C$ contains all of its limit points. Since $C=C$, Theorem \ref{trm:1.7} implies that $C\subset C$. Now suppose for the sake of contradiction that $C$ does not contain all of its limit points. Then there exists a point $p\in C$ that is a limit point of $C$ such that $p\notin C$. But we cannot have $p\in C$ and $p\notin C$, so it must be that the initial hypothesis was incorrect, meaning that $C$ does, in fact, contain all of its limit points.
    \end{proof}
\end{theorem}

\begin{theorem}\label{trm:4.3}
    A subset of $C$ containing a finite number of points is closed.
    \begin{proof}
        Let $A$ be a finite subset of $C$. To prove that $A$ is closed, Definition \ref{dfn:4.1} tells us that it will suffice to show that $A$ contains all of its limit points. But by Theorem \ref{trm:3.24}, $A$ has no limit points, so the statement "$A$ contains all of its limit points" is vacuously true.
    \end{proof}
\end{theorem}

\begin{definition}\label{dfn:4.4}
    Let $X$ be a subset of $C$. The \textbf{closure} of $X$ is the subset $\overline{X}$ of $C$ defined by
    \begin{equation*}
        \overline{X} = X\cup LP(X)
    \end{equation*}
\end{definition}

\begin{theorem}\label{trm:4.5}
    $X\subset C$ is closed if and only if $X=\overline{X}$.
    \begin{proof}
        Suppose first that $X$ is closed. To prove that $X=\overline{X}$, Definition \ref{dfn:4.4} tells us that it will suffice to show that $X=X\cup LP(X)$. To show this, Definition \ref{dfn:1.2} tells us that we must verify that every element $x$ of $X$ is an element of $X\cup LP(X)$ and vice versa. First, let $x$ be an arbitrary element of $X$. Then by Definition \ref{dfn:1.5}, $x\in X\cup LP(X)$, as desired. Now let $x$ be an arbitrary element of $X\cup LP(X)$. Then by Definition \ref{dfn:1.5}, $x\in X$ or $x\in LP(X)$. We divide into two cases. If $x\in X$, then we are done. If $x\in LP(X)$, then $x\in X$ as desired for the following reason: Since $X$ is closed by hypothesis, Definition \ref{dfn:4.1} implies that $X$ contains all of its limit points, i.e., for all $y\in LP(X)$, $y\in X$; this implication notably applies to the $x$ in question.\par
        Now suppose that $X=\overline{X}$. To prove that $X$ is closed, Definition \ref{dfn:4.1} tells us that it will suffice to show that $X$ contains all of its limit points. By Theorem \ref{trm:1.7}, $LP(X)\subset X\cup LP(X)$. This combined with the fact that $X=X\cup LP(X)$ (by Definition \ref{dfn:4.4}, since $X=\overline{X}$) implies that $LP(X)\subset X$. It follows by Definition \ref{dfn:1.3} that every element of $LP(X)$ is an element of $X$, i.e., every limit point of $X$ is an element of $X$, i.e., $X$ contains all of its limit points, as desired.
    \end{proof}
\end{theorem}

\begin{theorem}\label{trm:4.6}\marginnote{\emph{11/5:}}
    Let $X\subset C$. Then $\overline{X}=\overline{\overline{X}}$.
    \begin{lemma*}
        If $p$ is an element of $LP(LP(X))$, then $p$ is an element of $LP(X)$.
        \begin{proof}
            Let $p$ be an arbitrary element of $LP(LP(X))$. To prove that $p\in LP(X)$, Definition \ref{dfn:3.13} tells us that it will suffice to show that for all regions $R$ containing $p$, $R\cap(X\setminus\{p\})\neq\emptyset$. Let $R$ be an arbitrary region with $p\in R$ (we know that such a region exists because of Theorem \ref{trm:3.12}\footnote{This justification will not be supplied in similar cases beyond this point.}). Then since we know that $p\in LP(LP(X))$, we have by Theorem \ref{trm:3.26} that $R\cap LP(X)$ is infinite. Thus, we know that there exists an object $x\in R\cap LP(X)$ such that $x\neq p$. By Definition \ref{dfn:1.6}, it follows that $x\in R$ and $x\in LP(X)$. Since $x\in LP(X)$, Theorem \ref{trm:3.26} tells us that all regions containing $x$ (including $R$) have infinite intersection with $X$, i.e., $R\cap X$ is infinite. Consequently, $R\cap(X\setminus\{p\})$ is still infinite (since $\{p\}$ is finite), so $R\cap(X\setminus\{p\})\neq\emptyset$, as desired.
        \end{proof}
    \end{lemma*}
    \begin{proof}[Proof of Theorem \ref{trm:4.6}]
        To prove that $\overline{X}=\overline{\overline{X}}$, repeated applications of Definition \ref{dfn:4.4} tell us that it will suffice to show that
        \begin{equation*}
            X\cup LP(X) = (X\cup LP(X))\cup LP(X\cup LP(X))
        \end{equation*}
        To show this, Theorem \ref{trm:1.7a} tells us that it will suffice to verify the two statements
        \begin{align*}
            X\cup LP(X) &\subset (X\cup LP(X))\cup LP(X\cup LP(X))&
            (X\cup LP(X))\cup LP(X\cup LP(X)) &\subset X\cup LP(X)
        \end{align*}
        By Theorem \ref{trm:1.7b}, the left statement above is true. Consequently, all that's left at this point is to verify the right statement. To do so, Definition \ref{dfn:1.3} tells us that it will suffice to demonstrate that every point $p\in(X\cup LP(X))\cup LP(X\cup LP(X))$ is an element of $X\cup LP(X)$. Let's begin.\par
        Let $p$ be an arbitrary element of $(X\cup LP(X))\cup LP(X\cup LP(X))$. Then by Definition \ref{dfn:1.5}, $p\in X\cup LP(X)$ or $p\in LP(X\cup LP(X))$. We divide into two cases. Suppose first that $p\in X\cup LP(X)$. Since this is actually exactly what we want to prove, we are done. Now suppose that $p\in LP(X\cup LP(X))$. Then we have by Theorem \ref{trm:3.20} that $p\in LP(X)$ or $p\in LP(LP(X))$. We divide into two cases again. If $p\in LP(X)$, then by Definition \ref{dfn:1.5}, $p\in X\cup LP(X)$, and we are done. On the other hand, if $p\in LP(LP(X))$, then by the lemma, $p\in LP(X)$. Therefore, as before, $p\in X\cup LP(X)$, and we are done.
    \end{proof}
\end{theorem}

\begin{corollary}\label{cly:4.7}
    Let $X\subset C$. Then $\overline{X}$ is closed.
    \begin{proof}
        By Theorem \ref{trm:4.6}, $\overline{X}=\overline{\overline{X}}$. Thus, if we let $Y=\overline{X}$, we know that $Y=\overline{Y}$. But by Theorem \ref{trm:4.5}, this implies that $Y$, i.e., $\overline{X}$, is closed, as desired.
    \end{proof}
\end{corollary}

\begin{definition}\label{dfn:4.8}
    A subset $G$ of a continuum $C$ is \textbf{open} if its complement $C\setminus G$ is closed.
\end{definition}

\begin{theorem}\label{trm:4.9}
    The sets $\emptyset$ and $C$ are open.
    \begin{proof}
        We will address the two sets individually. to prove that $\emptyset$ is open, Definition \ref{dfn:4.8} tells us that it will suffice to show that $C\setminus\emptyset$ is closed. But $C\setminus\emptyset=C$, and by Theorem \ref{trm:4.2}, $C$ is closed. The proof is symmetric for $C$.
    \end{proof}
\end{theorem}

\begin{theorem}\label{trm:4.10}
    Let $G\subset C$. Then $G$ is open if and only if for all $x\in G$, there exists a region $R$ such that $x\in R$ and $R\subset G$.
    \begin{proof}
        To prove that $G$ is open if and only if for all $x\in G$, there exists a region $R$ such that $x\in R$ and $R\subset G$, we will take a similar approach to the proof of Theorem \ref{trm:3.20}. Indeed, to prove the dual implications "if there exists a region $R$ such that $x\in R$ and $R\subset G$ for all $x\in G$, then $G$ is open" and "if $G$ is open, then for all $x\in G$, there exists a region $R$ such that $x\in R$ and $R\subset G$," we will prove the first implication directly and the second one by contrapositive. Let's begin.\par
        Suppose first that for all $x\in G$, there exists a region $R$ such that $x\in R$ and $R\subset G$. To prove that $G$ is open, Definition \ref{dfn:4.8} tells us that it will suffice to confirm that $C\setminus G$ is closed. To confirm this, Definition \ref{dfn:4.1} tells us that it will suffice to show that $C\setminus G$ contains all of its limit points. Suppose for the sake of contradiction that for some limit point $p$ of $C\setminus G$, $p\notin C\setminus G$. Since $p\notin C\setminus G$, Definition \ref{dfn:1.11} tells us that $p\notin C$ or $p\in G$. But we must have $p\in C$, so necessarily $p\in G$. It follows by the hypothesis that there exists a region $R$ such that $p\in R$ and $R\subset G$. The fact that $R\subset G$ implies that $R\cap(C\setminus G)=\emptyset$. Consequently, $R\cap((C\setminus G)\setminus\{p\})=\emptyset$. But this implies by Definition \ref{dfn:3.13} that $p$ is not a limit point of $C\setminus G$, a contradiction. Therefore, $C\setminus G$ contains all of its limit points, as desired.\par
        Now suppose that for some $x\in G$, there does not exist a region $R$ such that $x\in R$ and $R\subset G$. To prove that $G$ is not open, Definition \ref{dfn:4.8} tells us that it will suffice to show that its complement $C\setminus G$ is not closed. To show this, Definition \ref{dfn:4.1} tells us that it will suffice to verify that $C\setminus G$ does not contain all of its limit points, i.e., it will suffice to find some limit point of $C\setminus G$ that is not an element of this set. Consider the $x\in G$ introduced by the hypothesis; we will prove that this $x$ is the desired limit point of $C\setminus G$ that is also not an element of $C\setminus G$. By Definition \ref{dfn:1.11}, $x\in G$ implies $x\notin C\setminus G$, so all that's left at this point is to prove that $x\in LP(C\setminus G)$. To do this, Definition \ref{dfn:3.13} tells us that it will suffice to demonstrate that for all $R$ with $x\in R$, $R\cap((C\setminus G)\setminus\{x\})\neq\emptyset$. Let $R$ be an arbitrary region with $x\in R$. By the hypothesis, $R\not\subset G$, so Definition \ref{dfn:1.3} implies that there exists some $y\in R$ such that $y\notin G$. Since $y\notin G$, we know two things: First, $y\neq x$ (since $x\in G$ and $y$ cannot be both an element of and not an element of $G$) and second, $y\in C\setminus G$ (see Definition \ref{dfn:1.11}). Consequently, we have $y\in R$ and $y\in C\setminus G$, so by Definition \ref{dfn:1.6}, $y\in R\cap(C\setminus G)$. It follows since $y\neq x$ by Definition \ref{dfn:1.11} that $y\in R\cap((C\setminus G)\setminus\{x\})$. Therefore, by Definition \ref{dfn:1.8}, $R\cap((C\setminus G)\setminus\{x\})\neq\emptyset$, as desired.
    \end{proof}
\end{theorem}

\begin{corollary}\label{cly:4.11}
    Every region $R$ is open. Every complement of a region $C\setminus R$ is closed.
    \begin{proof}
        Let $R$ be an arbitrary region. Clearly, for all $x\in R$, there exists a region (namely $R$) such that $x\in R$ and $R\subset R$. Thus, by Theorem \ref{trm:4.10}, $R$ is open, as desired. It follows by Definition \ref{dfn:4.8} that $C\setminus R$ is closed, as desired.
    \end{proof}
\end{corollary}

\begin{corollary}\label{cly:4.12}
    Let $G\subset C$. Then $G$ is open if and only if for all $x\in G$, there exists a subset $V\subset G$ such that $x\in V$ and $V$ is open.
    \begin{proof}
        Suppose first that $G$ is open. Then by Theorem \ref{trm:4.10}, for all $x\in G$, there exists a region $R$ such that $x\in R$ and $R\subset G$. Additionally, by Corollary \ref{cly:4.11}, each of these regions $R$ is open. Thus, $R$ is the desired open subset $V\subset G$ with $x\in V$.\par
        Now suppose that for all $x\in G$, there exists a subset $V\subset G$ such that $x\in V$ and $V$ is open. To prove that $G$ is open, Theorem \ref{trm:4.10} tells us that it will suffice to show that for all $x\in G$, there exists a region $R$ such that $x\in R$ and $R\subset G$. Let $x$ be an arbitrary element of $G$. By the hypothesis, we know that $x\in V$ where $V$ is an open subset of $G$. It follows by Theorem \ref{trm:4.10} that there exists a region $R$ such that $x\in R$ and $R\subset V$. But by subset transitivity, for any $R$, $R\subset G$. Thus, there exists a region $R$ such that $x\in R$ and $R\subset G$, as desired.
    \end{proof}
\end{corollary}

\begin{corollary}\label{cly:4.13}
    Let $a\in C$. Then the sets $\{x\in C\mid x<a\}$ and $\{x\in C\mid a<x\}$ are open.
    \begin{proof}
        We divide into two cases.\par
        First, consider the set $\{x\in C\mid x<a\}$, which we will henceforth call $G$ for ease of use. To prove that $G$ is open, Theorem \ref{trm:4.10} tells us that it will suffice to show that for all $y\in G$, there exists a region $R$ such that $y\in R$ and $R\subset G$. Let $y$ be an arbitrary element of $G$. By Axiom \ref{axm:3.3} and Definition \ref{dfn:3.3}, we can choose a $z<y$. Since we also have $y<a$ (by the definition of $G$ and the fact that $y\in G$), Definitions \ref{dfn:3.6} and \ref{dfn:3.10} assert that $y\in\underline{za}$. Now we must prove that $\underline{za}\subset G$, which Definition \ref{dfn:1.3} tells us we can do by showing that every element of $\underline{za}$ is an element of $G$. But since every element of $\underline{za}$ is less than $a$ (and greater than $z$, but this is not relevant) by Definitions \ref{dfn:3.10} and \ref{dfn:3.6}, they \emph{are} all elements of $G$ by the definition of $G$, as desired.\par
        The proof is symmetric in the other case.
    \end{proof}
\end{corollary}

\begin{theorem}\label{trm:4.14}
    Let $G$ be a nonempty open set. Then $G$ is the union of a collection of regions.
    \begin{proof}
        Since $G$ is an open set, Theorem \ref{trm:4.10} implies that for all $x\in G$, there exists a region $R_x$ such that $x\in R_x$ and $R_x\subset G$. Since we do not know that this $R_x$ is unique and we need to choose one $R_x$ for each of the (possibly infinite) $x\in G$, we invoke the axiom of choice to identify a single $R_x$ with each $x$ with which we can work. Thus, we can create the set $\mathcal{R}=\{R_x\mid x\in G\}$ and define the union of its elements $\bigcup_{x\in G}R_x$ by Definition \ref{dfn:1.13}. We now prove that $\bigcup_{x\in G}R_x=G$, which will suffice to show that $G$ can be described as a collection of regions.\par
        To prove that $\bigcup_{x\in G}R_x=G$, Definition \ref{dfn:1.2} tells us that it will suffice to show that every element $y\in\bigcup_{x\in G}R_x$ is an element of $G$ and vice versa. Suppose first that $y$ is an arbitrary element of $\bigcup_{x\in G}R_x$. Then by Definition \ref{dfn:1.13}, $y\in R_x$ for some $x\in G$. Thus, since $R_x\subset G$ by definition, Definition \ref{dfn:1.3} implies that $y\in G$. Now suppose that $y$ is an arbitrary element of $G$. Then $y\in R_y$, so Definition \ref{dfn:1.13} implies that $y\in\bigcup_{x\in G}R_x$, as desired.
    \end{proof}
\end{theorem}

\begin{exercise}\label{exr:4.15}\marginnote{11/10:}
    Do there exist subsets $X\subset C$ that are neither open nor closed?
    \begin{proof}
        Yes. Consider the continuum $\Q$ (see Exercise \ref{exr:3.9}) and the subset $\underline{12}\cup\{2\}\subset\Q$.\par\smallskip
        To prove that $\underline{12}\cup\{2\}$ is not closed, Definition \ref{dfn:4.1} tells us that it will suffice to show that there exists a limit point of $\underline{12}\cup\{2\}$ that is not an element of $\underline{12}\cup\{2\}$. The object $1\in\Q$ is the desired point, as will now be proven.\par
        To show that $1\notin\underline{12}\cup\{2\}$, Definition \ref{dfn:1.5} tells us that it will suffice to confirm that $1\notin\underline{12}$ and $1\notin\{2\}$. Suppose first for the sake of contradiction that $1\in\underline{12}$. Then by Definitions \ref{dfn:3.10} and \ref{dfn:3.6}, $1<1$ and $1<2$. But we also have $1=1$, contradicting Definition \ref{dfn:3.1}. Now suppose for the sake of contradiction that $1\in\{2\}$. Then $1=2$, a contradiction.\par
        To show that $1\in LP(\underline{12}\cup\{2\})$, Definition \ref{dfn:3.13} tells us that it will suffice to show that for all regions $\underline{ab}$ containing 1, $\underline{ab}\cap((\underline{12}\cup\{2\})\setminus\{1\})\neq\emptyset$. Let $\underline{ab}$ be an arbitrary region containing 1. Then by Definitions \ref{dfn:3.10} and \ref{dfn:3.6}, $a<1$ and $1<b$. We now divide into two cases ($b<2$ and $b\geq 2$). Suppose first that $b<2$. Since $1<b$, Additional Exercise \ref{axr:3.1} implies that there exists a point $r\in\Q$ such that $1<r<b$. Thus, by consecutive applications of Definition \ref{dfn:3.1}, $a<1$ and $1<r$ imply $a<r$, and $r<b$ and $b<2$ imply $r<2$. Consequently, by consecutive applications of Definitions \ref{dfn:3.6} and \ref{dfn:3.10}, $a<r$ and $r<b$ imply $r\in\underline{ab}$, and $1<r$ and $r<2$ imply $r\in\underline{12}$. Thus, by Definition \ref{dfn:1.5}, $r\in\underline{12}\cup\{2\}$. Additionally, since $1<r$, we have by Definition \ref{dfn:3.1} that $r\neq 1$, i.e., $r\notin\{1\}$. Thus, the facts that $r\in\underline{12}\cup\{2\}$ and $r\notin\{1\}$ imply by Definition \ref{dfn:1.11} that $r\in(\underline{12}\cup\{2\})\setminus\{1\}$. Therefore, since $r\in\underline{ab}$ and $r\in(\underline{12}\cup\{2\})\setminus\{1\}$, Definition \ref{dfn:1.6} implies that $r\in\underline{ab}\cap((\underline{12}\cup\{2\})\setminus\{1\})$, so by Definition \ref{dfn:1.8}, $\underline{ab}\cap((\underline{12}\cup\{2\})\setminus\{1\})\neq\emptyset$, as desired. Now suppose that $b\geq 2$. Then since $a<1<1.5<2\leq b$, we have by consecutive applications of Definitions \ref{dfn:3.6} and \ref{dfn:3.10} that $1.5\in\underline{ab}$ and $1.5\in\underline{12}$. It follows from the latter result and Definition \ref{dfn:1.5} that $1.5\in\underline{12}\cup\{2\}$. Since we also have $1.5\neq 1$, i.e., $1.5\notin\{1\}$, Definition \ref{dfn:1.11} tells us that $1.5\in(\underline{12}\cup\{2\})\setminus\{1\}$. Thus, since $1.5\in\underline{ab}$ and $1.5\in(\underline{12}\cup\{2\})\setminus\{1\}$, Definition \ref{dfn:1.6} implies that $1.5\in\underline{ab}\cap((\underline{12}\cup\{2\})\setminus\{1\})$. Therefore, by Definition \ref{dfn:1.8}, $\underline{ab}\cap((\underline{12}\cup\{2\})\setminus\{1\})\neq\emptyset$, as desired.\par\smallskip
        To prove that $\underline{12}\cup\{2\}$ is not open, Theorem \ref{trm:4.10} tells us that it will suffice to show that for some $x\in\underline{12}\cup\{2\}$, no region $\underline{ab}$ exists such that $x\in\underline{ab}$ and $\underline{ab}\subset\underline{12}\cup\{2\}$. In other words, we must show that for some $x\in\underline{12}\cup\{2\}$, all regions $\underline{ab}$ with $x\in\underline{ab}$ satisfy $\underline{ab}\not\subset\underline{12}\cup\{2\}$. Let $\underline{ab}$ be an arbitrary region with $2\in\underline{ab}$. Then by Definitions \ref{dfn:3.10} and \ref{dfn:3.6}, $a<2<b$. Since $2<b$, we have by Additional Exercise \ref{axr:3.1} that there exists a point $r\in\Q$ such that $2<r<b$, i.e., $a<2<r<b$. It follows by Definitions \ref{dfn:3.6} and \ref{dfn:3.10} that $r\in\underline{ab}$. Additionally, since $2<r$, we have $2\neq r$, i.e., $r\notin\{2\}$ and we have by Definitions \ref{dfn:3.6} and \ref{dfn:3.10} that $r\notin\underline{12}$. Consequently, by Definition \ref{dfn:1.5}, $r\notin\underline{12}\cup\{2\}$. Therefore, we have found an element of $\underline{ab}$ (namely $r$) that is not an element of $\underline{12}\cup\{2\}$, so by Definition \ref{dfn:1.3}, $\underline{ab}\not\subset\underline{12}\cup\{2\}$, as desired.
    \end{proof}
\end{exercise}

\begin{theorem}\label{trm:4.16}
    Let $\{X_\lambda\}$ be an arbitrary collection of closed subsets of a continuum $C$. Then the intersection $\bigcap_\lambda X_\lambda$ is closed.
    \begin{proof}
        To prove that $\bigcap_\lambda X_\lambda$ is closed, Definition \ref{dfn:4.1} tells us that it will suffice to show that every limit point of $\bigcap_\lambda X_\lambda$ is an element of $\bigcap_\lambda X_\lambda$. Let $p$ be an arbitrary limit point of $\bigcap_\lambda X_\lambda$. Then since $\bigcap_\lambda X_\lambda\subset X_\lambda$ for all $\lambda$, we have by Theorem \ref{trm:3.14} that $p$ is a limit point of every $X_\lambda$. Consequently, since every $X_\lambda$ is closed by hypothesis, Definition \ref{dfn:4.1} implies that $p\in X_\lambda$ for all $\lambda$. Thus, by Definition \ref{dfn:1.13}, $p\in\bigcap_\lambda X_\lambda$, as desired.
    \end{proof}
\end{theorem}

\begin{theorem}\label{trm:4.17}\marginnote{\emph{11/12:}}
    Let $G_1,\dots,G_n$ be a finite collection of open subsets of a continuum $C$. Then the intersection $G_1\cap\cdots\cap G_n$ is open.
    \begin{proof}
        To prove that $\bigcap_{k=1}^nG_k$ is open, Theorem \ref{trm:4.10} tells us that it will suffice to show that for all $x\in\bigcap_{k=1}^nG_k$, there exists a region containing $x$ that is a subset of $\bigcap_{k=1}^nG_k$. Let $x$ be an arbitrary element of $\bigcap_{k=1}^nG_k$. Then by Definition \ref{dfn:1.13}, $x\in G_k$ for all $k\in[n]$. Consequently, since all $G_k$ are open by hypothesis, Theorem \ref{trm:4.10} implies that for every $G_k$, there exists a region $R_k$ such that $x\in R_k$ and $R_k\subset G_k$. Since every $R_k$ has $x$ in common, Corollary \ref{cly:3.19} applies and implies that $\bigcap_{k=1}^nR_k$ is a region containing $x$. Additionally, since every $R_k\subset G_k$, we have by Script \ref{sct:1} that $\bigcap_{k=1}^nR_k\subset\bigcap_{k=1}^nG_k$. Therefore, $\bigcap_{k=1}^nR_k$ is a region containing $x$ that is a subset of $\bigcap_{k=1}^nG_k$, as desired.
    \end{proof}
\end{theorem}

\begin{corollary}\label{cly:4.18}
    Let $\{G_\lambda\}$ be an arbitrary collection of open subsets of a continuum $C$. Then the union $\bigcup_\lambda G_\lambda$ is open. Let $X_1,\dots,X_n$ be a finite collection of closed subsets of a continuum $C$. Then the union $X_1\cup\cdots\cup X_n$ is closed.
    \begin{proof}
        To prove that $\bigcup_\lambda G_\lambda$ is open, Theorem \ref{trm:4.10} tells us that it will suffice to show that for all $x\in\bigcup_\lambda G_\lambda$, there exists a region $R$ such that $x\in R$ and $R\subset\bigcup_\lambda G_\lambda$. Let $x$ be an arbitrary element of $\bigcup_\lambda G_\lambda$. Then $x\in G_\lambda$ for some $\lambda$. It follows by the fact that $G_\lambda$ is open by hypothesis and Theorem \ref{trm:4.10} that there exists a region $R$ such that $x\in R$ and $R\subset G_\lambda$. But since $G_\lambda\subset\bigcup_\lambda G_\lambda$, too, we have by subset transitivity that $R$ is a region such that $x\in R$ and $R\subset\bigcup_\lambda G_\lambda$, as desired.\par
        To prove that $\bigcup_{i=1}^nX_i$ is closed, Definition \ref{dfn:4.8} tells us that it will suffice to show that $C\setminus\bigcup_{i=1}^nX_i$ is open. To show this, Theorem \ref{trm:1.14} tells us that it will suffice to confirm that $\bigcap_{i=1}^n\left( C\setminus X_i \right)$ is open. But since $X_1,\dots,X_n$ is a finite collection of closed subsets of $C$, Definition \ref{dfn:4.8} implies that $C\setminus X_1,\dots,C\setminus X_n$ is a finite collection of open subsets of $C$. Therefore, by Theorem \ref{trm:4.17}, $\bigcap_{i=1}^n\left( C\setminus X_i \right)$, as desired.
    \end{proof}
\end{corollary}

\begin{exercise}\label{exr:4.19}
    Is it necessarily the case that the intersection of an infinite number of open sets is open? Is it possible to construct an infinite collection of open sets whose intersection is not open? Equivalently, is it possible to construct an infinite collection of closed sets whose union is not closed?
    \begin{proof}
        It is possible to construct an infinite collection of closed sets whose union is not closed. Consider the continuum $\Q$ and the region $\underline{12}\subset\Q$. By a similar argument to that used in Exercise \ref{exr:4.15}, $1\in LP(\underline{12})$ but $1\notin\underline{12}$. Thus, by Definition \ref{dfn:4.1}, $\underline{12}$ is not closed. Additionally, since $\Q$ is countable (see Theorem \ref{trm:2.11}) and $\underline{12}\subset\Q$ is infinite, we have by Exercise \ref{exr:1.37} that $\underline{12}$ is countable. Consequently, by Definition \ref{dfn:1.35} and \ref{dfn:1.28}, there exists a bijection $f:\N\to\underline{12}$. It follows that $\bigcup_{n\in\N}\{f(n)\}$ is well-defined, and we will now prove that $\bigcup_{n\in\N}\{f(n)\}=\underline{12}$.\par
        To prove that $\bigcup_{n\in\N}\{f(n)\}=\underline{12}$, Definition \ref{dfn:1.2} tells us that it will suffice to show that every element $x\in\bigcup_{n\in\N}\{f(n)\}$ is an element of $\underline{12}$ and vice versa. Let $x$ be an arbitrary element of $\bigcup_{n\in\N}\{f(n)\}$. Then $x\in\{f(n)\}$, i.e., $x=f(n)$ for some $n\in\N$. Thus, by the definition of $f$, $x\in\underline{12}$, as desired. Now let $x$ be an arbitrary element of $\underline{12}$. Then since $f$ is a bijection, $x=f(n)$ for some $n\in\N$. It follows that $x\in\{f(n)\}$ for some $n\in\N$, i.e., $x\in\bigcup_{n\in\N}\{f(n)\}$, as desired.\par
        Since $\bigcup_{n\in\N}\{f(n)\}=\underline{12}$ is not closed but every $\{f(n)\}$ is closed by Theorem \ref{trm:4.3}, then $\bigcup_{n\in\N}\{f(n)\}$ is an infinite collection of closed sets whose union is not closed. Equivalently, we have that $C\setminus\bigcup_{n\in\N}\{f(n)\}$ is not open (see Definition \ref{dfn:4.8}) and $C\setminus\bigcup_{n\in\N}\{f(n)\}=\bigcap_{n\in\N}\left( C\setminus\{f(n)\} \right)$ is the intersection of an infinite number of open sets (see Theorem \ref{trm:4.3} and Definition \ref{dfn:4.8}), as desired.
    \end{proof}
\end{exercise}

\begin{corollary}\label{cly:4.20}\marginnote{\emph{11/17:}}
    Let $G\subset C$ be nonempty. Then $G$ is open if and only if $G$ is the union of a collection of regions.
    \begin{proof}
        Suppose first that $G$ is open. Then by Theorem \ref{trm:4.14}, $G$ is the union of a collection of regions.\par
        Now suppose that $G$ is the union of a collection of regions. Then by Corollary \ref{cly:4.11}, $G$ is the union of a collection of open subsets of $C$. Therefore, by Corollary \ref{cly:4.18}, $G$ is open.
    \end{proof}
\end{corollary}

\begin{corollary}\label{cly:4.21}
    If $\underline{ab}$ is a region in $C$, then $\ext\underline{ab}$ is open.
    \begin{proof}
        By Corollary \ref{cly:4.13}, $\{x\in C\mid x<a\}$ and $\{x\in C\mid b<x\}$ are open. It follows by Corollary \ref{cly:4.18} that $\{x\in C\mid x<a\}\cup\{x\in C\mid b<x\}$ is open. Therefore, since $\{x\in C\mid x<a\}\cup\{x\in C\mid b<x\}=\ext\underline{ab}$ by Lemma \ref{lem:3.16}, $\ext\underline{ab}$ is open, as desired.
    \end{proof}
\end{corollary}

\begin{definition}\label{dfn:4.22}
    Let $C$ be a continuum. We say that $C$ is \textbf{disconnected} if it may be written as $C=A\cup B$, where $A$ and $B$ are disjoint, non-empty, open sets. $C$ is \textbf{connected} if it is not disconnected.
\end{definition}

\begin{exercise}\label{exr:4.23}
    Let $C$ be a connected continuum and $a\in C$. Prove that $C\setminus\{a\}$ is a disconnected continuum.
    \begin{proof}
        To prove that $C\setminus\{a\}$ is a disconnected continuum, Definition \ref{dfn:4.22} tells us that it will suffice to show that $C\setminus\{a\}=A\cup B$ where $A$ and $B$ are disjoint, non-empty, open sets. Let $A=\{x\in C\setminus\{a\}\mid x<a\}$ and let $B=\{x\in C\setminus\{a\}\mid a<x\}$. We now work our way through the conditions.\par
        To show that $C\setminus\{a\}=A\cup B$, Definition \ref{dfn:1.2} tells us that it will suffice to confirm that every element $y\in C\setminus\{a\}$ is an element of $A\cup B$ and vice versa. Suppose first that $y\in C\setminus\{a\}$. Then $y\in C$ and $y\notin\{a\}$, i.e., $y\neq a$. Thus, we have by Definition \ref{dfn:3.1} that $y<a$ or $y>a$ (we cannot have $y=a$ since we have already established that $y\neq a$). We divide into two cases. If $y<a$, then since we also know that $y\in C\setminus\{a\}$, we have $y\in\{x\in C\setminus\{a\}\mid x<a\}$, i.e., $y\in A$, i.e., $y\in A\cup B$, as desired. On the other hand, if $y>a$, then for similar reasons to the other case, $y\in B$, so $y\in A\cup B$, as desired. Now suppose that $y\in A\cup B$. Then $y\in A$ or $y\in B$. We divide into two cases. If $y\in A$, then we have by the definition of $A$ that $y\in C\setminus\{a\}$. The proof is symmetric for the other case.\par
        To show that $A$ and $B$ are disjoint, Definition \ref{dfn:1.9} tells us that it will suffice to verify that $A\cap B=\emptyset$. Suppose for the sake of contradiction that $y\in A\cap B$. Then $y\in A$ and $y\in B$. It follows by the definitions of $A$ and $B$ that $y<a$ and $a<y$. But this contradicts Definition \ref{dfn:3.1}. Therefore, no object $y$ is an element of $A\cap B$, so by Definition \ref{dfn:1.8}, $A\cap B=\emptyset$, as desired.\par
        To show that $A$ and $B$ are non-empty, Definition \ref{dfn:1.8} tells us that it will suffice to find an object in $A$ and an object in $B$. We divide into two cases. By Axiom \ref{axm:3.3} and Definition \ref{dfn:3.3}, there exists an object $y\in C$ such that $y<a$. It follows from the latter condition and Definition \ref{dfn:3.1} that $y\neq a$, i.e., $y\notin\{a\}$. Thus, since $y\in C$ and $y\notin\{a\}$, $y\in C\setminus\{a\}$. This result combined with the fact that $y<a$ implies by the definition of $A$ that $y\in A$, as desired. The proof is symmetric in the other case.\par
        To prove that $A$ and $B$ are open, we need look no farther than Corollary \ref{cly:4.13}.
    \end{proof}
\end{exercise}

\begin{exercise}\label{exr:4.24}
    Must every realization of a continuum be disconnected? Think about the realizations of the continuum from Exercise \ref{exr:3.27}. Are they connected/disconnected?
    \begin{proof}
        No. For example, $\Q$ from Exercise \ref{exr:3.27} is connected, while $\Z$ and $\Q\setminus\underline{01}$ are disconnected.
    \end{proof}
\end{exercise}




\end{document}