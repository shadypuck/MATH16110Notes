\documentclass[../main.tex]{subfiles}

\pagestyle{main}
\renewcommand{\chaptermark}[1]{\markboth{\chaptername\ \thechapter}{#1}}
\setcounter{chapter}{3}
\externaldocument{../main}

\begin{document}




\chapter{The Topology of a Continuum}
\section{Journal}
\begin{definition}\label{dfn:4.1}\marginnote{11/3:}
    A subset of a continuum is \textbf{closed} if it contains all of its limit points.
\end{definition}

\begin{theorem}\label{trm:4.2}
    The sets $\emptyset$ and $C$ are closed.
    \begin{proof}
        We will address the two sets individually.\par
        To prove that $\emptyset$ is closed, Definition \ref{dfn:4.1} tells us that it will suffice to show that $\emptyset\subset C$ and $\emptyset$ contains all of its limit points. By Exercise \ref{exr:1.10}, $\emptyset\subset C$. We now prove that $\emptyset$ has no limit points. Suppose for the sake of contradiction that some point $p\in C$ is a limit point of $\emptyset$. Then by Definition \ref{dfn:3.13}, for all regions $R$ with $p\in R$, $R\cap(\emptyset\setminus\{p\})\neq\emptyset$. But clearly, $R\cap(\emptyset\setminus\{p\})=R\cap\emptyset=\emptyset$, a contradiction. Therefore, since $\emptyset$ has no limit points, the statement "$\emptyset$ contains all of its limit points" is vacuously true.\par
        To prove that $C$ is closed, Definition \ref{dfn:4.1} tells us that it will suffice to show that $C\subset C$ and $C$ contains all of its limit points. Since $C=C$, Theorem \ref{trm:1.7} implies that $C\subset C$. Now suppose for the sake of contradiction that $C$ does not contain all of its limit points. Then there exists a point $p\in C$ that is a limit point of $C$ such that $p\notin C$. But we cannot have $p\in C$ and $p\notin C$, so it must be that the initial hypothesis was incorrect, meaning that $C$ does, in fact, contain all of its limit points.
    \end{proof}
\end{theorem}

\begin{theorem}\label{trm:4.3}
    A subset of $C$ containing a finite number of points is closed.
    \begin{proof}
        Let $A$ be a finite subset of $C$. To prove that $A$ is closed, Definition \ref{dfn:4.1} tells us that it will suffice to show that $A$ contains all of its limit points. But by Theorem \ref{trm:3.24}, $A$ has no limit points, so the statement "$A$ contains all of its limit points" is vacuously true.
    \end{proof}
\end{theorem}

\begin{definition}\label{dfn:4.4}
    Let $X$ be a subset of $C$. The \textbf{closure} of $X$ is the subset $\overline{X}$ of $C$ defined by
    \begin{equation*}
        \overline{X} = X\cup LP(X)
    \end{equation*}
\end{definition}

\begin{theorem}\label{trm:4.5}
    $X\subset C$ is closed if and only if $X=\overline{X}$.
    \begin{proof}
        Suppose first that $X$ is closed. To prove that $X=\overline{X}$, Definition \ref{dfn:4.4} tells us that it will suffice to show that $X=X\cup LP(X)$. To show this, Definition \ref{dfn:1.2} tells us that we must verify that every element $x$ of $X$ is an element of $X\cup LP(X)$ and vice versa. First, let $x$ be an arbitrary element of $X$. Then by Definition \ref{dfn:1.5}, $x\in X\cup LP(X)$, as desired. Now let $x$ be an arbitrary element of $X\cup LP(X)$. Then by Definition \ref{dfn:1.5}, $x\in X$ or $x\in LP(X)$. We divide into two cases. If $x\in X$, then we are done. If $x\in LP(X)$, then $x\in X$ as desired for the following reason: Since $X$ is closed by hypothesis, Definition \ref{dfn:4.1} implies that $X$ contains all of its limit points, i.e., for all $y\in LP(X)$, $y\in X$; this implication notably applies to the $x$ in question.\par
        Now suppose that $X=\overline{X}$. To prove that $X$ is closed, Definition \ref{dfn:4.1} tells us that it will suffice to show that $X$ contains all of its limit points. By Theorem \ref{trm:1.7}, $LP(X)\subset X\cup LP(X)$. This combined with the fact that $X=X\cup LP(X)$ (by Definition \ref{dfn:4.4}, since $X=\overline{X}$) implies that $LP(X)\subset X$. It follows by Definition \ref{dfn:1.3} that every element of $LP(X)$ is an element of $X$, i.e., every limit point of $X$ is an element of $X$, i.e., $X$ contains all of its limit points, as desired.
    \end{proof}
\end{theorem}

% \begin{itemize}
%     \item Let $p$ be an arbitrary point of $C$ such that $p\notin LP(X)$.
%     \item Definition \ref{dfn:3.13}: There exists a region $R$ with $p\in R$ such that $R\cap(X\setminus\{p\})=\emptyset$.
%     \item Let $R=\underline{ab}$ where $a,b\in C$, and let $q$ be an arbitrary point of $\underline{ab}$ such that $q\neq p$.
%     \item We divide into two cases (recall that $q=p$ is covered by the definition of $p$).
%     \item Case 1 ($q<p$):
%     \begin{itemize}
%         \item Consider the region $\underline{ap}$.
%         \item Prove two preliminary results.
%         \item $q\in\underline{ap}$:
%         \begin{itemize}
%             \item To prove this, Definitions \ref{dfn:3.10} and \ref{dfn:3.6}: show that $a<q$ and $q<p$.
%             \item We already know $q<p$ by hypothesis.
%             \item Definitions \ref{dfn:3.10} and \ref{dfn:3.6}: $q\in\underline{ab} \Longrightarrow a<q\wedge q<b$.
%         \end{itemize}
%         \item $\underline{ap}\subset R$:
%         \begin{itemize}
%             \item To prove this, Definition \ref{dfn:1.3}: show that every element $x\in\underline{ap}$ is an element of $R$.
%             \item Let $x$ be an arbitrary point of $\underline{ap}$.
%             \item Definitions \ref{dfn:3.10} and \ref{dfn:3.6}: $a<x$ and $x<p$.
%             \item Definitions \ref{dfn:3.10} and \ref{dfn:3.6}: $p\in R \Longrightarrow p\in\underline{ab} \Longrightarrow a<p\wedge p<b$.
%             \item Definitions \ref{dfn:3.1}: $x<p\wedge p<b \Longrightarrow x<b$.
%             \item Definitions \ref{dfn:3.6} and \ref{dfn:3.10}: $a<x\wedge x<b \Longrightarrow x\in\underline{ab} \Longrightarrow x\in R$.
%         \end{itemize}
%         \item Suppose (contradiction): there exists $x\in\underline{ap}\cap(X\setminus\{q\})$.
%         \item Definition \ref{dfn:1.6}: $x\in\underline{ap}$ and $x\in X\setminus\{q\}$.
%         \item Definition \ref{dfn:1.3}: $x\in\underline{ap}\wedge\underline{ap}\subset R \Longrightarrow x\in R$.
%         \item Definitions \ref{dfn:3.10} and \ref{dfn:3.6}: $x\in\underline{ap} \Longrightarrow a<x\wedge x<p$.
%         \item Definition \ref{dfn:3.1}: $x<p \Longrightarrow x\neq p \Longrightarrow x\notin\{p\}$.
%         \item Definition \ref{dfn:1.11}: $x\in X\setminus\{q\} \Longrightarrow x\in X\wedge x\notin\{q\}$.
%         \item Definition \ref{dfn:1.11}: $x\in X\wedge x\notin\{p\} \Longrightarrow x\in X\setminus\{p\}$.
%         \item Definition \ref{dfn:1.6}: $x\in R\wedge x\in X\setminus\{p\} \Longrightarrow x\in R\cap(X\setminus\{p\})$.
%         \item Definition \ref{dfn:1.2}: $x\in R\cap(X\setminus\{p\})\wedge R\cap(X\setminus\{p\})=\emptyset \Longrightarrow x\in\emptyset$.
%         \item Contradicts Definition \ref{dfn:1.8}.
%     \end{itemize}
%     \item Case 2 ($p<q$): The proof is symmetric.
%     \item Therefore, for all $q\in R$ (the objects equal to, less than, and greater than $p$), there exists a region $S$ with $q\in S$ (recall that we proved $q\in\underline{ap}$ in the case we treated, and $p\in R$ by definition) such that $S\cap(X\setminus\{q\})=\emptyset$.
%     \item Definition \ref{dfn:3.13}: For all $q\in R$, $q\notin LP(X)$.
%     \item Suppose (contradiction): there exists $x\in R\cap LP(X)$.
%     \item Definition \ref{dfn:1.6}: $x\in R$ and $x\in LP(X)$.
%     \item Contradicts $y\in R \Longrightarrow y\notin LP(X)$.
%     \item Definition \ref{dfn:1.8}: $R\cap LP(X)=\emptyset$.
%     \item Therefore, $R\cap(LP(X)\setminus\{p\})=\emptyset$.
%     \item Definition \ref{dfn:1.13}: $p\notin LP(LP(X))$.
% \end{itemize}

\begin{theorem}\label{trm:4.6}\marginnote{\emph{11/5:}}
    Let $X\subset C$. Then $\overline{X}=\overline{\overline{X}}$.
    \begin{lemma*}
        If $p$ is an element of $LP(LP(X))$, then $p$ is an element of $LP(X)$.
        \begin{proof}
            We will prove the claim by contrapositive. Because of the complexity of the argument used, a short outline follows. Essentially, we let $p\in C$ such that $p\notin LP(X)$. We show that this implies that there exists a region $R$ of $C$ with $p\in R$ such that for every point $q\in R$, $q\notin LP(X)$. This will imply that $R\cap LP(X)=\emptyset$, i.e., $R\cap(LP(X)\setminus\{p\})=\emptyset$, which means that $p$ is not an element of $LP(LP(X))$. Let's begin.\par\medskip
            Let $p$ be an arbitrary point of $C$ such that $p\notin LP(X)$. Then by Definition \ref{dfn:3.13}, there exists a region $\underline{ab}$ (where $a,b\in C$) with $p\in\underline{ab}$ such that $\underline{ab}\cap(X\setminus\{p\})=\emptyset$. Now let $q$ be an arbitrary point of $\underline{ab}$ such that $q\neq p$. To prove that $q\notin LP(X)$, we divide into two cases ($q<p$ and $p<q$; recall that $q=p$ is covered by the definition of $p$, which directly asserts that $p\notin LP(X)$). Note that if no $q\in\underline{ab}$ exists such that $q\neq p$, then the statement "$q\notin LP(X)$" is vacuously true.\par\smallskip
            Suppose first that $q<p$. Since $p\in\underline{ab}$, we have by Definitions \ref{dfn:3.10} and \ref{dfn:3.6} that $a<p$ and $p<b$. It follows from the former result and Definition \ref{dfn:3.10} that we the region $\underline{ap}$ is well defined. During our treatment of this case, we will spend a great deal of time examining this region. However, before we begin in earnest, we will prove two preliminary results (that $q\in\underline{ap}$ and that $\underline{ap}\subset\underline{ab}$).\par
            To prove that $q\in\underline{ap}$, Definitions \ref{dfn:3.10} and \ref{dfn:3.6} tell us that it will suffice to show that $a<q$ and $q<p$. We already know that $q<p$ by hypothesis, and since $q\in\underline{ab}$ by definition, Definitions \ref{dfn:3.10} and \ref{dfn:3.6} imply that $a<q$ (and $q<b$, but this is unimportant), as desired.\par
            To prove that $\underline{ap}\subset\underline{ab}$, Definition \ref{dfn:1.3} tells us that it will suffice to show that every element $x\in\underline{ap}$ is an element of $\underline{ab}$. Let $x$ be an arbitrary point of $\underline{ap}$. It follows by Definitions \ref{dfn:3.10} and \ref{dfn:3.6} that $a<x$ and $x<p$. By Definitions \ref{dfn:3.10} and \ref{dfn:3.6} again, the fact that $p\in\underline{ab}$ implies that $a<p$ and $p<b$. Thus, since $x<p$ and $p<b$, Definition \ref{dfn:3.1} asserts that $x<b$. Therefore, since $a<x$ and $x<b$, Definitions \ref{dfn:3.6} and \ref{dfn:3.10} imply that $x\in\underline{ab}$, as desired.\par
            With these two claims proven, we may now begin in earnest. Suppose for the sake of contradiction that there exists an object $x\in\underline{ap}\cap(X\setminus\{q\})$. Then by Definition \ref{dfn:1.6}, $x\in\underline{ap}$ and $x\in X\setminus\{q\}$. We investigate each result in turn. Using the former result and the fact that $\underline{ap}\subset\underline{ab}$, Definition \ref{dfn:1.3} tells us that $x\in\underline{ab}$ (this is an important result; remember it). Additionally, $x\in\underline{ap}$ reveals via Definitions \ref{dfn:3.10} and \ref{dfn:3.6} that $a<x$ and $x<p$. By Definition \ref{dfn:3.1}, $x<p$ implies that $x\neq p$, i.e., that $x\notin\{p\}$ (this is another important result). This concludes our investigation of the first result. With respect to the latter result, Definition \ref{dfn:1.11} implies that $x\in X$ (this is our final important result) and $x\notin\{q\}$. We now combine the three important results into a whole that leads to a contradiction. First off, since $x\in X$ and $x\notin\{p\}$, Definition \ref{dfn:1.11} tells us that $x\in X\setminus\{p\}$. This combined with the fact that $x\in\underline{ab}$ implies by Definition \ref{dfn:1.6} that $x\in\underline{ab}\cap(X\setminus\{p\})$. Consequently, since we also know from early on in this proof that $\underline{ab}\cap(X\setminus\{p\})=\emptyset$, we have by Definition \ref{dfn:1.2} that $x\in\emptyset$. But this contradicts Definition \ref{dfn:1.8}. Therefore, we have that $\underline{ap}\cap(X\setminus\{q\})=\emptyset$ for all $q<p$.\par\smallskip
            The proof is symmetric if $p<q$.\par\medskip
            Therefore, for all $q\in\underline{ab}$ (the objects equal to, less than, and greater than $p$), there exists a region $S$ with $q\in S$ (recall that we proved $q\in\underline{ap}$ in the case we treated, and $p\in\underline{ab}$ by definition) such that $S\cap(X\setminus\{q\})=\emptyset$. It follows by the contrapositive of Definition \ref{dfn:3.13} that for all $q\in\underline{ab}$, $q\notin LP(X)$.\par\medskip
            We are now very close to being done. To wrap it up, suppose for the sake of contradiction that there exists an object $x\in\underline{ab}\cap LP(X)$. Then by Definition \ref{dfn:1.6}, $x\in\underline{ab}$ and $x\in LP(X)$. But this contradicts the previously proven implication that if $x\in\underline{ab}$, we must also have $x\notin LP(X)$. Therefore, no object $x$ is an element of $\underline{ab}\cap LP(X)$, so by Definition \ref{dfn:1.8}, $\underline{ab}\cap LP(X)=\emptyset$. Consequently, $\underline{ab}\cap(LP(X)\setminus\{p\})=\emptyset$. Therefore, by the contrapositive of Definition \ref{dfn:1.13}, $p\notin LP(LP(X))$, as desired.
        \end{proof}
    \end{lemma*}
    \begin{proof}[Proof of Theorem \ref{trm:4.6}]
        To prove that $\overline{X}=\overline{\overline{X}}$, repeated applications of Definition \ref{dfn:4.4} tell us that it will suffice to show that
        \begin{equation*}
            X\cup LP(X) = (X\cup LP(X))\cup LP(X\cup LP(X))
        \end{equation*}
        To show this, Theorem \ref{trm:1.7a} tells us that it will suffice to verify the two statements
        \begin{align*}
            X\cup LP(X) &\subset (X\cup LP(X))\cup LP(X\cup LP(X))&
            (X\cup LP(X))\cup LP(X\cup LP(X)) &\subset X\cup LP(X)
        \end{align*}
        By Theorem \ref{trm:1.7b}, the left statement above is true. Consequently, all that's left at this point is to verify the right statement. To do so, Definition \ref{dfn:1.3} tells us that it will suffice to demonstrate that every point $p\in(X\cup LP(X))\cup LP(X\cup LP(X))$ is an element of $X\cup LP(X)$. Let's begin.\par
        Let $p$ be an arbitrary element of $(X\cup LP(X))\cup LP(X\cup LP(X))$. Then by Definition \ref{dfn:1.5}, $p\in X\cup LP(X)$ or $p\in LP(X\cup LP(X))$. We divide into two cases. Suppose first that $p\in X\cup LP(X)$. Since this is actually exactly what we want to prove, we are done. Now suppose that $p\in LP(X\cup LP(X))$. Then we have by Theorem \ref{trm:3.20} that $p\in LP(X)$ or $p\in LP(LP(X))$. We divide into two cases again. If $p\in LP(X)$, then by Definition \ref{dfn:1.5}, $p\in X\cup LP(X)$, and we are done. On the other hand, if $p\in LP(LP(X))$, then by the lemma, $p\in LP(X)$. Therefore, as before, $p\in X\cup LP(X)$, and we are done.
    \end{proof}
\end{theorem}

\begin{corollary}\label{cly:4.7}
    Let $X\subset C$. Then $\overline{X}$ is closed.
    \begin{proof}
        By Theorem \ref{trm:4.6}, $\overline{X}=\overline{\overline{X}}$. Thus, if we let $Y=\overline{X}$, we know that $Y=\overline{Y}$. But by Theorem \ref{trm:4.5}, this implies that $Y$, i.e., $\overline{X}$, is closed, as desired.
    \end{proof}
\end{corollary}

\begin{definition}\label{dfn:4.8}
    A subset $G$ of a continuum $C$ is \textbf{open} if its complement $C\setminus G$ is closed.
\end{definition}

\begin{theorem}\label{trm:4.9}
    The sets $\emptyset$ and $C$ are open.
    \begin{proof}
        We will address the two sets individually. to prove that $\emptyset$ is open, Definition \ref{dfn:4.8} tells us that it will suffice to show that $C\setminus\emptyset$ is closed. But $C\setminus\emptyset=C$, and by Theorem \ref{trm:4.2}, $C$ is closed. The proof is symmetric for $C$.
    \end{proof}
\end{theorem}




\end{document}