\documentclass[../main.tex]{subfiles}

\pagestyle{main}
\renewcommand{\chaptermark}[1]{\markboth{\chaptername\ \thechapter}{#1}}
\setcounter{chapter}{3}

\begin{document}




\chapter{The Topology of a Continuum}
\section{Journal}
\begin{definition}\label{dfn:4.1}\marginnote{11/3:}
    A subset of a continuum is \textbf{closed} if it contains all of its limit points.
\end{definition}

\begin{theorem}\label{trm:4.2}
    The sets $\emptyset$ and $C$ are closed.
    \begin{proof}
        We will address the two sets individually.\par
        To prove that $\emptyset$ is closed, Definition \ref{dfn:4.1} tells us that it will suffice to show that $\emptyset\subset C$ and $\emptyset$ contains all of its limit points. By Exercise \ref{exr:1.10}, $\emptyset\subset C$. We now prove that $\emptyset$ has no limit points. Suppose for the sake of contradiction that some point $p\in C$ is a limit point of $\emptyset$. Then by Definition \ref{dfn:3.13}, for all regions $R$ with $p\in R$, $R\cap(\emptyset\setminus\{p\})\neq\emptyset$. But clearly, $R\cap(\emptyset\setminus\{p\})=R\cap\emptyset=\emptyset$, a contradiction. Therefore, since $\emptyset$ has no limit points, the statement "$\emptyset$ contains all of its limit points" is vacuously true.\par
        To prove that $C$ is closed, Definition \ref{dfn:4.1} tells us that it will suffice to show that $C\subset C$ and $C$ contains all of its limit points. Since $C=C$, Theorem \ref{trm:1.7} implies that $C\subset C$. Now suppose for the sake of contradiction that $C$ does not contain all of its limit points. Then there exists a point $p\in C$ that is a limit point of $C$ such that $p\notin C$. But we cannot have $p\in C$ and $p\notin C$, so it must be that the initial hypothesis was incorrect, meaning that $C$ does, in fact, contain all of its limit points.
    \end{proof}
\end{theorem}

\begin{theorem}\label{trm:4.3}
    A subset of $C$ containing a finite number of points is closed.
    \begin{proof}
        Let $A$ be a finite subset of $C$. To prove that $A$ is closed, Definition \ref{dfn:4.1} tells us that it will suffice to show that $A$ contains all of its limit points. But by Theorem \ref{trm:3.24}, $A$ has no limit points, so the statement "$A$ contains all of its limit points" is vacuously true.
    \end{proof}
\end{theorem}

\begin{definition}\label{dfn:4.4}
    Let $X$ be a subset of $C$. The \textbf{closure} of $X$ is the subset $\overline{X}$ of $C$ defined by
    \begin{equation*}
        \overline{X} = X\cup LP(X)
    \end{equation*}
\end{definition}

\begin{theorem}\label{trm:4.5}
    $X\subset C$ is closed if and only if $X=\overline{X}$.
    \begin{proof}
        Suppose first that $X$ is closed. To prove that $X=\overline{X}$, Definition \ref{dfn:4.4} tells us that it will suffice to show that $X=X\cup LP(X)$. To show this, Definition \ref{dfn:1.2} tells us that we must verify that every element $x$ of $X$ is an element of $X\cup LP(X)$ and vice versa. First, let $x$ be an arbitrary element of $X$. Then by Definition \ref{dfn:1.5}, $x\in X\cup LP(X)$, as desired. Now let $x$ be an arbitrary element of $X\cup LP(X)$. Then by Definition \ref{dfn:1.5}, $x\in X$ or $x\in LP(X)$. We divide into two cases. If $x\in X$, then we are done. If $x\in LP(X)$, then $x\in X$ as desired for the following reason: Since $X$ is closed by hypothesis, Definition \ref{dfn:4.1} implies that $X$ contains all of its limit points, i.e., for all $y\in LP(X)$, $y\in X$; this implication notably applies to the $x$ in question.\par
        Now suppose that $X=\overline{X}$. To prove that $X$ is closed, Definition \ref{dfn:4.1} tells us that it will suffice to show that $X$ contains all of its limit points. By Theorem \ref{trm:1.7}, $LP(X)\subset X\cup LP(X)$. This combined with the fact that $X=X\cup LP(X)$ (by Definition \ref{dfn:4.4}, since $X=\overline{X}$) implies that $LP(X)\subset X$. It follows by Definition \ref{dfn:1.3} that every element of $LP(X)$ is an element of $X$, i.e., every limit point of $X$ is an element of $X$, i.e., $X$ contains all of its limit points, as desired.
    \end{proof}
\end{theorem}

\begin{theorem}\label{trm:4.6}\marginnote{\emph{11/5:}}
    Let $X\subset C$. Then $\overline{X}=\overline{\overline{X}}$.
    \begin{lemma*}
        If $p$ is an element of $LP(LP(X))$, then $p$ is an element of $LP(X)$.
        \begin{proof}
            Let $p$ be an arbitrary element of $LP(LP(X))$. To prove that $p\in LP(X)$, Definition \ref{dfn:3.13} tells us that it will suffice to show that for all regions $R$ containing $p$, $R\cap(X\setminus\{p\})\neq\emptyset$. Let $R$ be an arbitrary region with $p\in R$ (we know that such a region exists because of Theorem \ref{trm:3.12}\footnote{This justification will not be supplied in similar cases beyond this point.}). Then since we know that $p\in LP(LP(X))$, we have by Theorem \ref{trm:3.26} that $R\cap LP(X)$ is infinite. Thus, we know that there exists an object $x\in R\cap LP(X)$ such that $x\neq p$. By Definition \ref{dfn:1.6}, it follows that $x\in R$ and $x\in LP(X)$. Since $x\in LP(X)$, Theorem \ref{trm:3.26} tells us that all regions containing $x$ (including $R$) have infinite intersection with $X$, i.e., $R\cap X$ is infinite. Consequently, $R\cap(X\setminus\{p\})$ is still infinite (since $\{p\}$ is finite), so $R\cap(X\setminus\{p\})\neq\emptyset$, as desired.
        \end{proof}
    \end{lemma*}
    \begin{proof}[Proof of Theorem \ref{trm:4.6}]
        To prove that $\overline{X}=\overline{\overline{X}}$, repeated applications of Definition \ref{dfn:4.4} tell us that it will suffice to show that
        \begin{equation*}
            X\cup LP(X) = (X\cup LP(X))\cup LP(X\cup LP(X))
        \end{equation*}
        To show this, Theorem \ref{trm:1.7a} tells us that it will suffice to verify the two statements
        \begin{align*}
            X\cup LP(X) &\subset (X\cup LP(X))\cup LP(X\cup LP(X))&
            (X\cup LP(X))\cup LP(X\cup LP(X)) &\subset X\cup LP(X)
        \end{align*}
        By Theorem \ref{trm:1.7b}, the left statement above is true. Consequently, all that's left at this point is to verify the right statement. To do so, Definition \ref{dfn:1.3} tells us that it will suffice to demonstrate that every point $p\in(X\cup LP(X))\cup LP(X\cup LP(X))$ is an element of $X\cup LP(X)$. Let's begin.\par
        Let $p$ be an arbitrary element of $(X\cup LP(X))\cup LP(X\cup LP(X))$. Then by Definition \ref{dfn:1.5}, $p\in X\cup LP(X)$ or $p\in LP(X\cup LP(X))$. We divide into two cases. Suppose first that $p\in X\cup LP(X)$. Since this is actually exactly what we want to prove, we are done. Now suppose that $p\in LP(X\cup LP(X))$. Then we have by Theorem \ref{trm:3.20} that $p\in LP(X)$ or $p\in LP(LP(X))$. We divide into two cases again. If $p\in LP(X)$, then by Definition \ref{dfn:1.5}, $p\in X\cup LP(X)$, and we are done. On the other hand, if $p\in LP(LP(X))$, then by the lemma, $p\in LP(X)$. Therefore, as before, $p\in X\cup LP(X)$, and we are done.
    \end{proof}
\end{theorem}

\begin{corollary}\label{cly:4.7}
    Let $X\subset C$. Then $\overline{X}$ is closed.
    \begin{proof}
        By Theorem \ref{trm:4.6}, $\overline{X}=\overline{\overline{X}}$. Thus, if we let $Y=\overline{X}$, we know that $Y=\overline{Y}$. But by Theorem \ref{trm:4.5}, this implies that $Y$, i.e., $\overline{X}$, is closed, as desired.
    \end{proof}
\end{corollary}

\begin{definition}\label{dfn:4.8}
    A subset $G$ of a continuum $C$ is \textbf{open} if its complement $C\setminus G$ is closed.
\end{definition}

\begin{theorem}\label{trm:4.9}
    The sets $\emptyset$ and $C$ are open.
    \begin{proof}
        We will address the two sets individually. to prove that $\emptyset$ is open, Definition \ref{dfn:4.8} tells us that it will suffice to show that $C\setminus\emptyset$ is closed. But $C\setminus\emptyset=C$, and by Theorem \ref{trm:4.2}, $C$ is closed. The proof is symmetric for $C$.
    \end{proof}
\end{theorem}

\begin{theorem}\label{trm:4.10}
    Let $G\subset C$. Then $G$ is open if and only if for all $x\in G$, there exists a region $R$ such that $x\in R$ and $R\subset G$.
    \begin{proof}
        To prove that $G$ is open if and only if for all $x\in G$, there exists a region $R$ such that $x\in R$ and $R\subset G$, we will take a similar approach to the proof of Theorem \ref{trm:3.20}. Indeed, to prove the dual implications "if there exists a region $R$ such that $x\in R$ and $R\subset G$ for all $x\in G$, then $G$ is open" and "if $G$ is open, then for all $x\in G$, there exists a region $R$ such that $x\in R$ and $R\subset G$," we will prove the first implication directly and the second one by contrapositive. Let's begin.\par
        Suppose first that for all $x\in G$, there exists a region $R$ such that $x\in R$ and $R\subset G$. To prove that $G$ is open, Definition \ref{dfn:4.8} tells us that it will suffice to confirm that $C\setminus G$ is closed. To confirm this, Definition \ref{dfn:4.1} tells us that it will suffice to show that $C\setminus G$ contains all of its limit points. Suppose for the sake of contradiction that for some limit point $p$ of $C\setminus G$, $p\notin C\setminus G$. Since $p\notin C\setminus G$, Definition \ref{dfn:1.11} tells us that $p\notin C$ or $p\in G$. But we must have $p\in C$, so necessarily $p\in G$. It follows by the hypothesis that there exists a region $R$ such that $p\in R$ and $R\subset G$. The fact that $R\subset G$ implies that $R\cap(C\setminus G)=\emptyset$. Consequently, $R\cap((C\setminus G)\setminus\{p\})=\emptyset$. But this implies by Definition \ref{dfn:3.13} that $p$ is not a limit point of $C\setminus G$, a contradiction. Therefore, $C\setminus G$ contains all of its limit points, as desired.\par
        Now suppose that for some $x\in G$, there does not exist a region $R$ such that $x\in R$ and $R\subset G$. To prove that $G$ is not open, Definition \ref{dfn:4.8} tells us that it will suffice to show that its complement $C\setminus G$ is not closed. To show this, Definition \ref{dfn:4.1} tells us that it will suffice to verify that $C\setminus G$ does not contain all of its limit points, i.e., it will suffice to find some limit point of $C\setminus G$ that is not an element of this set. Consider the $x\in G$ introduced by the hypothesis; we will prove that this $x$ is the desired limit point of $C\setminus G$ that is also not an element of $C\setminus G$. By Definition \ref{dfn:1.11}, $x\in G$ implies $x\notin C\setminus G$, so all that's left at this point is to prove that $x\in LP(C\setminus G)$. To do this, Definition \ref{dfn:3.13} tells us that it will suffice to demonstrate that for all $R$ with $x\in R$, $R\cap((C\setminus G)\setminus\{x\})\neq\emptyset$. Let $R$ be an arbitrary region with $x\in R$. By the hypothesis, $R\not\subset G$, so Definition \ref{dfn:1.3} implies that there exists some $y\in R$ such that $y\notin G$. Since $y\notin G$, we know two things: First, $y\neq x$ (since $x\in G$ and $y$ cannot be both an element of and not an element of $G$) and second, $y\in C\setminus G$ (see Definition \ref{dfn:1.11}). Consequently, we have $y\in R$ and $y\in C\setminus G$, so by Definition \ref{dfn:1.6}, $y\in R\cap(C\setminus G)$. It follows since $y\neq x$ by Definition \ref{dfn:1.11} that $y\in R\cap((C\setminus G)\setminus\{x\})$. Therefore, by Definition \ref{dfn:1.8}, $R\cap((C\setminus G)\setminus\{x\})\neq\emptyset$, as desired.
    \end{proof}
\end{theorem}

\begin{corollary}\label{cly:4.11}
    Every region $R$ is open. Every complement of a region $C\setminus R$ is closed.
    \begin{proof}
        Let $R$ be an arbitrary region. Clearly, for all $x\in R$, there exists a region (namely $R$) such that $x\in R$ and $R\subset R$. Thus, by Theorem \ref{trm:4.10}, $R$ is open, as desired. It follows by Definition \ref{dfn:4.8} that $C\setminus R$ is closed, as desired.
    \end{proof}
\end{corollary}

\begin{corollary}\label{cly:4.12}
    Let $G\subset C$. Then $G$ is open if and only if for all $x\in G$, there exists a subset $V\subset G$ such that $x\in V$ and $V$ is open.
    \begin{proof}
        Suppose first that $G$ is open. Then by Theorem \ref{trm:4.10}, for all $x\in G$, there exists a region $R$ such that $x\in R$ and $R\subset G$. Additionally, by Corollary \ref{cly:4.11}, each of these regions $R$ is open. Thus, $R$ is the desired open subset $V\subset G$ with $x\in V$.\par
        Now suppose that for all $x\in G$, there exists a subset $V\subset G$ such that $x\in V$ and $V$ is open. To prove that $G$ is open, Theorem \ref{trm:4.10} tells us that it will suffice to show that for all $x\in G$, there exists a region $R$ such that $x\in R$ and $R\subset G$. Let $x$ be an arbitrary element of $G$. By the hypothesis, we know that $x\in V$ where $V$ is an open subset of $G$. It follows by Theorem \ref{trm:4.10} that there exists a region $R$ such that $x\in R$ and $R\subset V$. But by subset transitivity, for any $R$, $R\subset G$. Thus, there exists a region $R$ such that $x\in R$ and $R\subset G$, as desired.
    \end{proof}
\end{corollary}

\begin{corollary}\label{cly:4.13}
    Let $a\in C$. Then the sets $\{x\in C\mid x<a\}$ and $\{x\in C\mid a<x\}$ are open.
    \begin{proof}
        We divide into two cases.\par
        First, consider the set $\{x\in C\mid x<a\}$, which we will henceforth call $G$ for ease of use. To prove that $G$ is open, Theorem \ref{trm:4.10} tells us that it will suffice to show that for all $y\in G$, there exists a region $R$ such that $y\in R$ and $R\subset G$. Let $y$ be an arbitrary element of $G$. By Axiom \ref{axm:3.3} and Definition \ref{dfn:3.3}, we can choose a $z<y$. Since we also have $y<a$ (by the definition of $G$ and the fact that $y\in G$), Definitions \ref{dfn:3.6} and \ref{dfn:3.10} assert that $y\in\underline{za}$. Now we must prove that $\underline{za}\subset G$, which Definition \ref{dfn:1.3} tells us we can do by showing that every element of $\underline{za}$ is an element of $G$. But since every element of $\underline{za}$ is less than $a$ (and greater than $z$, but this is not relevant) by Definitions \ref{dfn:3.10} and \ref{dfn:3.6}, they \emph{are} all elements of $G$ by the definition of $G$, as desired.\par
        The proof is symmetric in the other case.
    \end{proof}
\end{corollary}

\begin{theorem}\label{trm:4.14}
    Let $G$ be a nonempty open set. Then $G$ is the union of a collection of regions.
    \begin{proof}
        Since $G$ is an open set, Theorem \ref{trm:4.10} implies that for all $x\in G$, there exists a region $R_x$ such that $x\in R_x$ and $R_x\subset G$. Thus, we can create the set $\mathcal{R}=\{R_x\mid x\in G\}$ and define its union $\bigcup_{x\in G}R_x$ by Definition \ref{dfn:1.13}. We now prove that $\bigcup_{x\in G}R_x=G$, which will suffice to show that $G$ can be described as a collection of regions.\par
        To prove that $\bigcup_{x\in G}R_x=G$, Definition \ref{dfn:1.2} tells us that it will suffice to show that every element $y\in\bigcup_{x\in G}R_x$ is an element of $G$ and vice versa. Suppose first that $y$ is an arbitrary element of $\bigcup_{x\in G}R_x$. Then by Definition \ref{dfn:1.13}, $y\in R_x$ for some $x\in G$. Thus, since $R_x\subset G$ by definition, Definition \ref{dfn:1.3} implies that $y\in G$. Now suppose that $y$ is an arbitrary element of $G$. Then $y\in R_y$, so Definition \ref{dfn:1.13} implies that $y\in\bigcup_{x\in G}R_x$, as desired.
    \end{proof}
\end{theorem}




\end{document}