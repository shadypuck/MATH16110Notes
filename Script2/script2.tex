\documentclass[../main.tex]{subfiles}

\pagestyle{main}
\renewcommand{\chaptermark}[1]{\markboth{\chaptername\ \thechapter}{#1}}
\setcounter{chapter}{1}
\externaldocument{../main}

\begin{document}




\chapter{The Rationals}\label{sct:2}
\section{Journal}
\begin{definition}\label{dfn:2.1}\marginnote{10/15:}
    Let $X$ be a set. A \textbf{relation} $R$ on $X$ is a subset of $X\times X$. The statement $(x,y)\in R$ is read "$x$ is related to $y$ by the relation $R$" and is often denoted $x\sim y$.\par\medskip
    \noindent A relation is \textbf{reflexive} if $x\sim x$ for all $x\in X$.\\
    A relation is \textbf{symmetric} if $y\sim x$ whenever $x\sim y$.\\
    A relation is \textbf{transitive} if $x\sim z$ whenever $x\sim y$ and $y\sim z$.\\
    A relation is an \textbf{equivalence relation} if it is reflexive, symmetric, and transitive.
\end{definition}
\smallskip

\begin{exercise}\label{exr:2.2}
    Determine which of the following are equivalence relations.
    \begin{enumerate}[label={\alph*)},ref={\theexercise\alph*}]
        \item \label{exr:2.2a}Any set $X$ with the relation $=$. So $x\sim y$ if and only if $x=y$.
        \begin{proof}
            To prove that the relation $=$ is reflexive, Definition \ref{dfn:2.1} tells us that it will suffice to show that $x\sim x$ for all $x\in X$. Clearly, $x=x$ for all $x\in X$. It follows by the definition of $=$ that $x\sim x$ for all $x\in X$. For symmetry, we must verify that $x\sim y$ implies $y\sim x$ for $x,y\in X$. Let $x\sim y$ for some $x,y\in X$. Consequently, by the definition of $=$, $x=y$. It follows that $y=x$, and thus that $y\sim x$. For transitivity, we must show that $x\sim y$ and $y\sim z$ imply that $x\sim z$ for $x,y,z\in X$. Let $x\sim y$ and $y\sim z$ for some $x,y,z\in X$. By the definition of $=$, $x\sim y$ and $y\sim z$ imply that $x=y$ and $y=z$, respectively. Thus, $x=y=z$, so $x=z$, meaning that $x\sim z$ by the definition of the relation $=$. Since the relation $=$ is reflexive, symmetric, and transitive, it is an equivalence relation.
        \end{proof}
        \item \label{exr:2.2b}$\Z$ with the relation $<$.
        \begin{proof}
            The relation $<$ is neither reflexive nor symmetric, although it is transitive. Since demonstrating that $<$ does not satisfy any one of the three properties proves that $<$ is not an equivalence relation, we shall arbitrarily choose to prove that $<$ is not reflexive. Consider $1\in\Z$, and note that $1=1$. Since $1=1$, $1\nless 1$ by the trichotomy. Thus, $1\nsim 1$ by the relation $<$, proving that $<$ is not reflexive for all $z\in Z$.
        \end{proof}
        \item \label{exr:2.2c}Any subset $X$ of $\Z$ with the relation $\leq$. So $x\sim y$ if and only if $x\leq y$.
        \begin{proof}
            Here, we demonstrate a failure of symmetry. Let $X=\{1,2\}$. Clearly, $X\subset\Z$. Now, $1\leq 2$, so $1\sim 2$ by the relation $\leq$, but $2\nleq 1$ so $2\nsim 1$. Thus, $x\sim y$ for $x,y\in X$ does not necessarily imply that $y\sim x$. It follows that $\leq$ is not an equivalence relation on \emph{any} subset of $\Z$.
        \end{proof}
        \item \label{exr:2.2d}$X=\Z$ with $x\sim y$ if and only if $y-x$ is divisible by 5.
        \begin{proof}
            To prove that the described relation is an equivalence relation, Definition \ref{dfn:2.1} tells us that we must verify that it is reflexive, symmetric, and transitive. To prove these properties, it will suffice to show that $x\sim x$ for all $x\in X$, $x\sim y$ implies $y\sim x$ for any $x,y\in X$, and $x\sim y$ and $y\sim z$ implies $x\sim z$ for any $x,y,z\in X$, respectively. Let's begin.\par
            To prove that $x\sim x$ for all $x\in X$, the definition of $\sim$ and Additional Exercise \ref{axr:0.8} tell us that it will suffice to show that $x-x=5a$ for an arbitrary $x\in X$ and some $a\in\Z$. Let $x$ be an arbitrary element of $X$. It follows that $x-x=0=5(0)$ where $0=a$ is clearly an element of $\Z$. In sum, $x-x=5a$ for an $a\in\Z$, as desired.\par
            To prove that $x\sim y$ implies that $y\sim x$, the definition of $\sim$ tells us that it will suffice to show that $x-y$ is divisible by 5 given that $y-x$ is so divisible for $x,y\in X$. Let $y-x$ be divisible by 5. It follows by Additional Exercise \ref{axr:0.8-ii} that $-1\cdot(y-x)$ is divisible by 5 (since $-1$ is clearly an integer). But since $-(y-x)=x-y$, this means that $x-y$ is divisible by 5, as desired.\par
            To prove that $x\sim y$ and $y\sim z$ imply that $x\sim z$, the definition of $\sim$ tells us that it will suffice to show that $z-x$ is divisible by 5 given that $y-x$ and $z-y$ are so divisible for $x,y,z\in X$. Let $y-x$ and $z-y$ be divisible by 5. It follows by Additional Exercise \ref{axr:0.8-i} that $(z-y)+(y-x)$ is divisible by 5. But since $(z-y)+(y-x)=z-x$, this means that $z-x$ is divisible by 5, as desired.
        \end{proof}
        \item \label{exr:2.2e}$X=\{(a,b)\mid a,b\in\Z,b\neq 0\}$ with the relation $\sim$ defined by $(a,b)\sim(c,d)\Longleftrightarrow ad=bc$.
        \begin{proof}
            Reflexivity: Let $(a,b)$ be an arbitrary element of $X$. Since $a,b\in\Z$ and integer multiplication is commutative, it is true that $ab=ba$. Therefore, by the definition of the relation $\sim$, $(a,b)\sim(a,b)$.\par
            Symmetry: Let $(a,b)\sim(c,d)$ for some $(a,b),(c,d)\in X$. By the definition of the relation $\sim$, $ad=bc$. Thus, $cb=da$ by the symmetry of $=$ (see Exercise \ref{exr:2.2a}) and the commutativity of integer multiplication. Therefore, by the definition of the relation $\sim$, $(c,d)\sim(a,b)$.\par
            Transitivity: Let $(a,b)\sim(c,d)$ and $(c,d)\sim(e,f)$ for some $(a,b),(c,d),(e,f)\in X$. By consecutive applications of the definition of $\sim$, $ad=bc$ and $cf=de$. We now divide into two cases ($c\neq 0$ and $c=0$; the reason for doing so will become clear later). Suppose first that $c\neq 0$. By the multiplicative property of equality, we can multiply an equal quantity to each side of $ad=bc$ and still preserve the equality. As such, we choose to multiply $cf=de$ to both sides, creating the equation $ad\cdot cf=bc\cdot de$. By the commutativity of multiplication, we have $afcd=becd$. Since $c\neq 0$ by assumption and $d\neq 0$ by the definition of $X$, $cd\neq 0$ and the cancellation law for multiplication applies, giving us $af=be$. Therefore, by the definition of the relation $\sim$, $(a,b)\sim(e,f)$. Now suppose that $c=0$. Consequently, $bc=0$, implying by the equality $ad=bc$ that $ad=0$. Thus, $a=0$ or $d=0$ (or both) by the zero product property. Since $d\neq 0$ by the definition of $X$ ($d$ is the second element in the ordered pair $(c,d)$), we must have $a=0$. A similar analysis can be performed on the equation $cf=de$ to show that $e=0$. Since $a=0$ and $e=0$, $af=0$ and $be=0$, implying by transitivity that $af=be$. Therefore, by the definition of the relation $\sim$, $(a,b)\sim(e,f)$.
        \end{proof}
    \end{enumerate}
\end{exercise}

\begin{remark}\label{rmk:2.3}
    A \textbf{partition} of a set is a collection of non-empty disjoint subsets whose union is the original set. Any equivalence relation on a set creates a partition of that set by collecting into subsets all of the elements that are equivalent (related) to each other. When the partition of a set arises from an equivalence relation in this manner, the subsets are referred to as \textbf{equivalence classes}.
\end{remark}

\begin{remark}\label{rmk:2.4}
    If we think of the set $X$ in Exercise \ref{exr:2.2e} as representing the collection of all fractions whose denominators are not zero, then the relation $\sim$ may be thought of as representing the equivalence of two fractions.
\end{remark}

\begin{definition}\label{dfn:2.5}
    As a set, the \textbf{rational numbers}, denoted $\Q$, are the equivalence classes in the set $X=\{(a,b)\mid a,b\in\Z,b\neq 0\}$ under the equivalence relation $\sim$ as defined in Exercise \ref{exr:2.2e}. If $(a,b)\in X$, we denote the equivalence class of this element as $\eqclass{a}{b}$. So
    \begin{equation*}
        \eqclass{a}{b} = \{(x_1,x_2)\in X\mid(x_1,x_2)\sim(a,b)\}
        = \{(x_1,x_2)\in X\mid x_1b=x_2a\}
    \end{equation*}
    Then,
    \begin{equation*}
        \Q = \left\{\eqclass{a}{b}\middle|(a,b)\in X\right\}
    \end{equation*}
\end{definition}

\begin{exercise}\label{exr:2.6}
    $\eqclass{a}{b}=\eqclass{a'}{b'}\Longleftrightarrow(a,b)\sim(a',b')$
    \begin{proof}
        Suppose first that $\eqclass{a}{b}=\eqclass{a'}{b'}$. Since $\eqclass{a}{b}\in\Q$, Definition \ref{dfn:2.5} implies that $(a,b)\in X$. It follows by Exercise \ref{exr:2.2e} that $(a,b)\sim(a,b)$. The last two results imply by Definition \ref{dfn:2.5} that $(a,b)\in\eqclass{a}{b}$. Consequently, set equality implies that $(a,b)\in\eqclass{a'}{b'}$. But by Definition \ref{dfn:2.5}, this means that $(a,b)\sim(a',b')$, as desired. Now suppose that $(a,b)\sim(a',b')$. To prove that $\eqclass{a}{b}=\eqclass{a'}{b'}$, Definition \ref{dfn:1.2} tells us that we must verify that every element of $\eqclass{a}{b}$ is an element of $\eqclass{a'}{b'}$ and vice versa. Let $(x_1,x_2)$ be an arbitrary element of $\eqclass{a}{b}$. It follows by Definition \ref{dfn:2.5} that $(x_1,x_2)\in X$ and that $(x_1,x_2)\sim(a,b)$. The latter result combined with the hypothesis that $(a,b)\sim(a',b')$ implies by the transitivity of $\sim$ (see Exercise \ref{exr:2.2e}) that $(x_1,x_2)\sim(a',b')$. This new finding coupled with the fact that $(x_1,x_2)\in X$ implies by Definition \ref{dfn:2.5} that $(x_1,x_2)\in\eqclass{a'}{b'}$, as desired. The proof is symmetric if we first let that $(x_1,x_2)$ be an arbitrary element of $\eqclass{a'}{b'}$.
    \end{proof}
\end{exercise}

\begin{definition}\label{dfn:2.7}
    We define the binary operations addition and multiplication on $\Q$ as follows. If $\eqclass{a}{b},\eqclass{c}{d}\in\Q$, then
    \begin{gather*}
        \eqclass{a}{b}\plusQ\eqclass{c}{d} = \eqclass{ad+bc}{bd}\\
        \eqclass{a}{b}\cdotQ\eqclass{c}{d} = \eqclass{ac}{bd}
    \end{gather*}
    We use the notation $\plusQ$ and $\cdotQ$ to represent addition and multiplication in $\Q$ so as to distinguish these operations from the usual addition ($+$) and multiplication ($\cdot$) in $\Z$.
\end{definition}

\begin{theorem}\label{trm:2.8}
    Addition in $\Q$ is well-defined. That is, if $(a,b)\sim(a',b')$ and $(c,d)\sim(c',d')$, then
    \begin{equation*}
        \eqclass{a}{b}\plusQ\eqclass{c}{d} = \eqclass{a'}{b'}\plusQ\eqclass{c'}{d'}
    \end{equation*}
    \begin{proof}
        By consecutive applications of the definition of $\sim$, we have from the hypotheses that
        \begin{align*}
            ab' &= ba'&
                cd' &= dc'
            \intertext{It follows by the multiplicative property of equality that}
            ab'dd' &= ba'dd'&
                bb'cd' &= bb'dc'
        \end{align*}
        The above two results can be combined via the additive property of equality, giving the following, which will be algebraically manipulated further.
        \begin{align*}
            ab'dd'+bb'cd' &= ba'dd'+bb'dc'\\
            adb'd'+bcb'd' &= bda'd'+bdb'c'\\
            (ad+bc)(b'd') &= (bd)(a'd'+b'c')
        \end{align*}
        The last line above implies by the definition of $\sim$ that $(ad+bc,bd)\sim(a'd'+b'c',b'd')$. It follows by Exercise \ref{exr:2.6} that
        \begin{align*}
            \eqclass{ad+bc}{bd} &= \eqclass{a'd'+b'c'}{b'd'}
            \intertext{Therefore, by two applications of Definition \ref{dfn:2.7},}
            \eqclass{a}{b}\plusQ\eqclass{c}{d} &= \eqclass{a'}{b'}\plusQ\eqclass{c'}{d'}
        \end{align*}
        as desired.
    \end{proof}
\end{theorem}

\begin{theorem}\label{trm:2.9}
    Multiplication in $\Q$ is well-defined. That is, if $(a,b)\sim(a',b')$ and $(c,d)\sim(c',d')$, then
    \begin{equation*}
        \eqclass{a}{b}\cdotQ\eqclass{c}{d} = \eqclass{a'}{b'}\cdotQ\eqclass{c'}{d'}
    \end{equation*}
    \begin{proof}
        By consecutive applications of the definition of $\sim$, we have from the hypotheses that $ab'=ba'$ and $cd'=dc'$. Multiplying these equations together, we have $ab'cd'=ba'dc'$. This can be algebraically rearranged into $(ac)(b'd')=(bd)(a'c')$. It follows by the definition of $\sim$ that $(ac,bd)\sim(a'c',b'd')$. But this implies by Exercise \ref{exr:2.6} that
        \begin{align*}
            \eqclass{ac}{bd} &= \eqclass{a'c'}{b'd'}
            \intertext{Consequently, by Definition \ref{dfn:2.7},}
            \eqclass{a}{b}\cdotQ\eqclass{c}{d} &= \eqclass{a'}{b'}\cdotQ\eqclass{c'}{d'}
        \end{align*}
        as desired.
    \end{proof}
\end{theorem}

\begin{theorem}\label{trm:2.10}\leavevmode
    \begin{enumerate}[label={\alph*\textup{)}},ref={\thetheorem\alph*}]
        \item \label{trm:2.10a}Commutativity of addition
        \begin{equation*}
            \eqclass{a}{b}\plusQ\eqclass{c}{d} = \eqclass{c}{d}\plusQ\eqclass{a}{b}\text{ for all }\eqclass{a}{b},\eqclass{c}{d}\in\Q
        \end{equation*}
        \begin{proof}
            By Definition \ref{dfn:2.7},
            \begin{align*}
                \eqclass{a}{b}\plusQ\eqclass{c}{d} &= \eqclass{ad+bc}{bd}
                \intertext{With integer algebra, we can rearrange the above expression into}
                &= \eqclass{cb+da}{db}\\
                \intertext{By Definition \ref{dfn:2.7} again, the above}
                &= \eqclass{c}{d}\plusQ\eqclass{a}{b}
            \end{align*}
        \end{proof}
        \item \label{trm:2.10b}Associativity of addition
        \begin{equation*}
            \left( \eqclass{a}{b}\plusQ\eqclass{c}{d} \right)\plusQ\eqclass{e}{f} = \eqclass{a}{b}\plusQ\left( \eqclass{c}{d}\plusQ\eqclass{e}{f} \right)\text{ for all }\eqclass{a}{b},\eqclass{c}{d},\eqclass{e}{f}\in\Q
        \end{equation*}
        \begin{proof}
            By consecutive applications of Definition \ref{dfn:2.7},
            \begin{align*}
                \left( \eqclass{a}{b}\plusQ\eqclass{c}{d} \right)\plusQ\eqclass{e}{f} &= \eqclass{ad+bc}{bd}\plusQ\eqclass{e}{f}\\
                &= \eqclass{(ad+bc)(f)+(bd)(e)}{(bd)(f)}
                \intertext{With integer algebra, we can rearrange the above as follows.}
                &= \eqclass{adf+bcf+bde}{bdf}\\
                &= \eqclass{(a)(df)+(b)(cf+de)}{(b)(df)}
                \intertext{Now apply Definition \ref{dfn:2.7} twice, again.}
                &=\eqclass{a}{b}\plusQ\left( \eqclass{cf+de}{df} \right)\\
                &=\eqclass{a}{b}\plusQ\left( \eqclass{c}{d}\plusQ\eqclass{e}{f} \right)
            \end{align*}
        \end{proof}
        \item \label{trm:2.10c}Existence of an additive identity
        \begin{equation*}
            \eqclass{a}{b}\plusQ\eqclass{0}{1} = \eqclass{a}{b}\text{ for all }\eqclass{a}{b}\in\Q
        \end{equation*}
        \begin{proof}
            Via Definition \ref{dfn:2.7} and integer algebra, we can show that
            \begin{align*}
                \eqclass{a}{b}\plusQ\eqclass{0}{1} &= \eqclass{a\cdot 1+b\cdot 0}{b\cdot 1}\\
                &= \eqclass{a}{b}
            \end{align*}
            as desired.
        \end{proof}
        \item \label{trm:2.10d}Existence of additive inverses
        \begin{equation*}
            \eqclass{a}{b}\plusQ\eqclass{-a}{b} = \eqclass{0}{1}\text{ for all }\eqclass{a}{b}\in\Q
        \end{equation*}
        \begin{proof}
            Through various application of Definition \ref{dfn:2.7} and integer algebra, we can show that
            \begin{align*}
                \eqclass{a}{b}\plusQ\eqclass{-a}{b} &= \eqclass{ab+b\cdot -a}{bb}\\
                &= \eqclass{ab-ab}{bb}\\
                &= \eqclass{0}{bb}
                \intertext{Since $0\cdot 1=0$ and $bb\cdot 0=0$, transitivity implies that $(0)(1)=(bb)(0)$. By the definition of $\sim$, this means that $(0,bb)\sim(0,1)$. It follows by Exercise \ref{exr:2.6} that the above equals the following, as desired.}
                &= \eqclass{0}{1}
            \end{align*}
        \end{proof}
        \item \label{trm:2.10e}Commutativity of multiplication
        \begin{equation*}
            \eqclass{a}{b}\cdotQ\eqclass{c}{d} = \eqclass{c}{d}\cdotQ\eqclass{a}{b}\text{ for all }\eqclass{a}{b},\eqclass{c}{d}\in\Q
        \end{equation*}
        \begin{proof}
            Via Definition \ref{dfn:2.7} and integer algebra, we can show that
            \begin{align*}
                \eqclass{a}{b}\cdotQ\eqclass{c}{d} &= \eqclass{ac}{bd}\\
                &= \eqclass{ca}{db}\\
                &= \eqclass{c}{d}\cdotQ\eqclass{a}{b}
            \end{align*}
            as desired.
        \end{proof}
        \item \label{trm:2.10f}Associativity of multiplication
        \begin{equation*}
            \left( \eqclass{a}{b}\cdotQ\eqclass{c}{d} \right)\cdotQ\eqclass{e}{f} = \eqclass{a}{b}\cdotQ\left( \eqclass{c}{d}\cdotQ\eqclass{e}{f} \right)\text{ for all }\eqclass{a}{b},\eqclass{c}{d},\eqclass{e}{f}\in\Q
        \end{equation*}
        \begin{proof}
            Through various application of Definition \ref{dfn:2.7} and integer algebra, we can show that
            \begin{align*}
                \left( \eqclass{a}{b}\cdotQ\eqclass{c}{d} \right)\cdotQ\eqclass{e}{f} &= \eqclass{ac}{bd}\cdotQ\eqclass{e}{f}\\
                &= \eqclass{(ac)(e)}{(bd)(f)}\\
                &= \eqclass{(a)(ce)}{(b)(df)}\\
                &=\eqclass{a}{b}\cdotQ\left( \eqclass{ce}{df} \right)\\
                &=\eqclass{a}{b}\cdotQ\left( \eqclass{c}{d}\cdotQ\eqclass{e}{f} \right)
            \end{align*}
            as desired.
        \end{proof}
        \item \label{trm:2.10g}Existence of a multiplicative identity
        \begin{equation*}
            \eqclass{a}{b}\cdotQ\eqclass{1}{1} = \eqclass{a}{b}\text{ for all }\eqclass{a}{b}\in\Q
        \end{equation*}
        \begin{proof}
            Via Definition \ref{dfn:2.7} and integer algebra, we can show that
            \begin{align*}
                \eqclass{a}{b}\cdotQ\eqclass{1}{1} &= \eqclass{a\cdot 1}{b\cdot 1}\\
                &= \eqclass{a}{b}
            \end{align*}
            as desired.
        \end{proof}
        \item \label{trm:2.10h}Existence of multiplicative inverses for nonzero elements
        \begin{equation*}
            \eqclass{a}{b}\cdotQ\eqclass{b}{a} = \eqclass{1}{1}\text{ for all }\eqclass{a}{b}\in\Q\text{ such that }\eqclass{a}{b}\neq\eqclass{0}{1}
        \end{equation*}
        \begin{lemma*}
            For all $\eqclass{a}{a}\in\Q$, $\eqclass{a}{a}=\eqclass{1}{1}$.
            \begin{proof}
                Since $(a)(1)=(a)(1)$, we have by the definition of $\sim$ that $(a,a)\sim(1,1)$. It follows by Exercise \ref{exr:2.6} that $\eqclass{a}{a}=\eqclass{1}{1}$, as desired.
            \end{proof}
        \end{lemma*}
        \begin{proof}
            By Definition \ref{dfn:2.7},
            \begin{align*}
                \eqclass{a}{b}\cdotQ\eqclass{b}{a} &= \eqclass{ab}{ba}
                \intertext{Since $ab=ba$, we have by the lemma that the above equals the following, as desired.}
                &= \eqclass{1}{1}
            \end{align*}
        \end{proof}
        \item \label{trm:2.10i}Distributivity
        \begin{equation*}
            \eqclass{a}{b}\cdotQ\left( \eqclass{c}{d}\plusQ\eqclass{e}{f} \right) = \left( \eqclass{a}{b}\cdotQ\eqclass{c}{d} \right)\plusQ\left( \eqclass{a}{b}\cdotQ\eqclass{e}{f} \right)\text{ for all }\eqclass{a}{b},\eqclass{c}{d},\eqclass{e}{f}\in\Q
        \end{equation*}
        \begin{proof}
            By Definition \ref{dfn:2.7} and integer algebra,
            \begin{align*}
                \eqclass{a}{b}\cdotQ\left( \eqclass{c}{d}\plusQ\eqclass{e}{f} \right) &= \eqclass{a}{b}\cdotQ\eqclass{cf+de}{df}\\
                &= \eqclass{a(cf+de)}{bdf}\\
                &= \eqclass{acf+ade}{bdf}
                \intertext{Use Theorem \ref{trm:2.10g}.}
                &= \eqclass{acf+ade}{bdf}\cdotQ\eqclass{1}{1}
                \intertext{Use the lemma from the proof of Theorem \ref{trm:2.10h}.}
                &= \eqclass{acf+ade}{bdf}\cdotQ\eqclass{b}{b}
                \intertext{Use various applications of Definition \ref{dfn:2.7} and integer algebra to finish.}
                &= \eqclass{(acf+ade)b}{(bdf)b}\\
                &= \eqclass{acfb+adeb}{bdfb}\\
                &= \eqclass{(ac)(bf)+(bd)(ae)}{(bd)(bf)}\\
                &= \eqclass{ac}{bd}\plusQ\eqclass{ae}{bf}\\
                &= \left( \eqclass{a}{b}\cdotQ\eqclass{c}{d} \right)\plusQ\left( \eqclass{a}{b}\cdotQ\eqclass{e}{f} \right)
            \end{align*}
            as desired.
        \end{proof}
    \end{enumerate}
\end{theorem}

\begin{theorem}\label{trm:2.11}\marginnote{\emph{10/20:}}
    $\Q$ is countable.
    \begin{lemma*}\leavevmode
        \begin{enumerate}[label={\alph*\textup{)}}]
            \item If there exists a surjection $g:B\to A$, then there exists an injection $f:A\to B$.
            \item The set $\Z\times(\Z\setminus\{0\})$ is countable.
        \end{enumerate}
        \begin{proof}[Proof of a]
            Let $f:A\to B$ be defined such that for all $a\in A$, $f(a)\in g^{-1}(\{a\})$\footnote{Note that we are not defining $f$ explicitly, but rather providing a rule that means that some matchings will not suffice to define $f$, namely ones for which $f(a)\notin g^{-1}(\{a\})$ for all $a\in A$.}. To prove that this condition is well-defined (i.e., there is no $a\in A$ such that $f(a)$ \emph{cannot} be an element of $g^{-1}(\{a\})$), we will show that for all $a\in A$, $g^{-1}(\{a\})$ contains at least one element of $B$. Let $a$ be an arbitrary element of $A$. Since $g$ is surjective, we know by Definition \ref{dfn:1.20} that there is a $b\in B$ such that $g(b)=a$. Let's consider this $b$ more closely. As an element of $B$ satisfying the condition that $g(b)=a$, $b$ is naturally an element of the set $\{b'\in B\mid g(b')=a\}$. Clearly, this set is equivalent to $\{b'\in B\mid g(b')\in\{a\}\}$, so $b$ is also an element of this new set. But by Definition \ref{dfn:1.18}, this set is equal to $g^{-1}(\{a\})$. Thus, $b\in B$ and $b\in g^{-1}(\{a\})$, as desired.\par
            To prove that $f$ is injective, Definition \ref{dfn:1.20} tells us that it will suffice to show that $f(a)=f(a')$ implies that $a=a'$. Let $f(a)=f(a')$. It follows by the condition imposed on $f$ that $f(a)\in g^{-1}(\{a\})$ and $f(a')\in g^{-1}(\{a'\})$. With respect to the latter case, the fact that $f(a)=f(a')$ also implies that $f(a)\in g^{-1}(\{a'\})$. Because $f(a)\in g^{-1}(\{a\})$ and $f(a)\in g^{-1}(\{a'\})$, Definition \ref{dfn:1.18} tells us that $g(f(a))\in\{a\}$ and $g(f(a))\in\{a'\}$, respectively. Consequently, $g(f(a))=a$ and $g(f(a))=a'$, respectively. Since $g$ is a function, Definition \ref{dfn:1.16} implies that $g(f(a))$ is a unique, well-defined object, so $a=g(f(a))=a'$, i.e., $a=a'$, as desired.
        \end{proof}
        \begin{proof}[Proof of b]
            By Exercise \ref{exr:1.36}, $\Z$ is countable, i.e.\footnote{Let $A$ be a set (such as $\Z$). Technically, Definition \ref{dfn:1.35} must be invoked to move from "$A$ is countable" to "$A$ is in bijective correspondence with $\N$," and Definition \ref{dfn:1.28} must be invoked to move from "$A$ is in bijective correspondence with $\N$" to "there exists a bijection $f:A\to\N$." However, as we are no longer in Script \ref{sct:1}, such justifications will not be supplied beyond this footnote.}, there exists a bijection $f_1:\Z\to\N$. Since $\Z\setminus\{0\}\subset\Z$, Exercise \ref{exr:1.37} implies that $\Z\setminus\{0\}$ is countable, i.e., there exists a bijection $f_2:\Z\setminus\{0\}\to\N$. Now let $f:\Z\times(\Z\setminus\{0\})\to\N\times\N$ be defined by $f(a,b)=(f_1(a),f_2(b))$. To prove that $f$ is a function, Definition \ref{dfn:1.16} tells us that we must show that for every $(a,b)\in\Z\times(\Z\setminus\{0\})$, there is a unique $(c,d)\in\N\times\N$ such that $f(a,b)=(c,d)$. Let $(a,b)$ be an arbitrary element of $\Z\times(\Z\setminus\{0\})$. Then by Definition \ref{dfn:1.15}, $a\in\Z$ and $b\in\Z\setminus\{0\}$. Thus, by the definitions of $f_1$ and $f_2$, $f_1(a)$ and $f_2(b)$ are defined objects and elements of $\N$. Consequently, Definition \ref{dfn:1.15} implies that $(f_1(a),f_2(a))\in\N\times\N$. Since $f(a,b)=(f_1(a),f_2(b))$ by the definition of $f$, it follows that $(f_1(a),f_2(b))$ is \emph{an} element of $\N\times\N$ to which $f$ maps $(a,b)$. On the uniqueness of this object, suppose that $f(a,b)=(c,d)$ and $f(a,b)=(c',d')$ for some $(a,b)\in\Z\times(\Z\setminus\{0\})$. By the definition of $f$, this implies that $(f_1(a),f_2(b))=(c,d)$ and $(f_1(a),f_2(b))=(c',d')$. Thus, by multiple applications of Definition \ref{dfn:1.15}, $f_1(a)=c$, $f_1(a)=c'$, $f_2(b)=d$, and $f_2(b)=d'$. But since $f_1$ and $f_2$ are both functions, Definition \ref{dfn:1.16} implies that $f_1(a)$ and $f_2(b)$ are both unique, well-defined objects. Consequently, transitivity applies and implies that $c=f_1(a)=c'$ and $d=f_2(b)=d'$. It follows by Definition \ref{dfn:1.15} once again that $(c,d)=(c',d')$, as desired.\par
            To prove that $f$ is injective, Definition \ref{dfn:1.20} tells us that we must verify that $f(a,b)=f(a',b')$ implies that $(a,b)=(a',b')$. Let $f(a,b)=f(a',b')$. By the definition of $f$, $(f_1(a),f_2(b))=(f_1(a'),f_2(b'))$. Thus, by Definition \ref{dfn:1.15}, $f_1(a)=f_1(a')$ and $f_2(b)=f_2(b')$. Consequently, by the injectivity of $f_1$ and $f_2$ (which follows from their respective bijectivity by Definition \ref{dfn:1.20}), $a=a'$ and $b=b'$. Therefore, by Definition \ref{dfn:1.15} once again, $(a,b)=(a',b')$.\par
            By Exercise \ref{exr:1.39}, $\N\times\N$ is countable, i.e., there exists a bijection $g:\N\times\N\to\N$. Since $g$ is bijective, Definition \ref{dfn:1.20} implies that it is injective. Thus, by Proposition \ref{prp:1.26}, $g\circ f:\Z\times(\Z\setminus\{0\})\to\N$ (which Definition \ref{dfn:1.25} guarantees exists) is injective.\par
            Since there exists an injection $g\circ f:\Z\times(\Z\setminus\{0\})\to\N$ where $\N$ is clearly countable and $\Z\times(\Z\setminus\{0\})$ is clearly infinite, Exercise \ref{exr:1.38} implies that $\Z\times(\Z\setminus\{0\})$ is countable, as desired.
        \end{proof}
    \end{lemma*}
    \begin{proof}[Proof of Theorem \ref{trm:2.11}]
        Let $g:\Z\times(\Z\setminus\{0\})\to\Q$ be defined by $g(a,b)=\eqclass{a}{b}$. For $g$ to be a function as defined by Definition \ref{dfn:1.16}, $g$ must map every ordered pair $(a,b)\in\Z\times(\Z\setminus\{0\})$ to a unique element $\eqclass{a}{b}$ in $\Q$. Let $(a,b)$ be an arbitrary element of $\Z\times(\Z\setminus\{0\})$. By definition, $g$ clearly maps $(a,b)$ to only one (i.e., a unique) object, namely the equivalence class $\eqclass{a}{b}$. But we must still show that this $\eqclass{a}{b}$ is an element of $\Q$ (note that this is not immediately obvious as equivalence classes such as $\eqclass{0}{0}$ [which is still a well-defined equivalence class, just an empty one] are not elements of $\Q$). For $\eqclass{a}{b}$ to be an element of $\Q$, Definition \ref{dfn:2.5} tells us that it will suffice to show that $(a,b)\in X$. For $(a,b)$ to be an element of $X$, Exercise \ref{exr:2.2e} asserts that it will suffice to show that $a,b\in\Z$ and $b\neq 0$. But since $(a,b)\in\Z\times(\Z\setminus\{0\})$ by assumption, Definition \ref{dfn:1.15} tells us that $a\in\Z$ and $b\in\Z\setminus\{0\}$. Expounding on the latter result, Definition \ref{dfn:1.11} tells us that $b\in\Z$ and $b\notin\{0\}$, i.e., $b\in\Z$ and $b\neq 0$. Combining the last three results, we have that $a,b\in\Z$ and $b\neq 0$, as desired.\par
        To prove that $g$ is surjective, Definition \ref{dfn:1.20} tells us that we must verify that for all $\eqclass{a}{b}\in\Q$, there exists an $(a,b)\in\Z\times(\Z\setminus\{0\})$ such that $g(a,b)=\eqclass{a}{b}$. Let $\eqclass{a}{b}$ be an arbitrary element of $\Q$. By Definition \ref{dfn:2.5}, $(a,b)\in X$. Thus, by Exercise \ref{exr:2.2e}, $a,b\in\Z$ and $b\neq 0$. Since $b\in\Z$ and $b\neq 0$, i.e., $b\in\Z$ and $b\notin\{0\}$, Definition \ref{dfn:1.11} tells us that $b\in\Z\setminus\{0\}$. To recap, $a\in\Z$ and $b\in\Z\setminus\{0\}$. But by Definition \ref{dfn:1.15}, this implies that $(a,b)\in\Z\times(\Z\setminus\{0\})$. With regards to this $(a,b)$, we have by the definition of $g$ that $g(a,b)=\eqclass{a}{b}$, as desired.\par
        Since there exists a surjection $g:\Z\times(\Z\setminus\{0\})\to\Q$, we have by Lemma (a) that there exists an injection $f:\Q\to\Z\times(\Z\setminus\{0\})$. Thus, we have an injection $f:\Q\to\Z\times(\Z\setminus\{0\})$ where $\Z\times(\Z\setminus\{0\})$ is countable (by Lemma (b)) and $\Q$ is clearly infinite. By Exercise \ref{exr:1.38}, this means that $\Q$ is countable.
    \end{proof}
\end{theorem}




\end{document}