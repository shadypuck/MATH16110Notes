\documentclass[../main.tex]{subfiles}

\pagestyle{main}
\renewcommand{\chaptermark}[1]{\markboth{\chaptername\ \thechapter}{#1}}
\setcounter{chapter}{4}
\setcounter{axiom}{3}

\begin{document}




\chapter{Connectedness and Boundedness}\label{sct:5}
\section{Journal}
\begin{axiom}\label{axm:5.4}\marginnote{\emph{11/19:}}
    A continuum is connected.
\end{axiom}

\begin{theorem}\label{trm:5.1}
    The only subsets of a continuum $C$ that are both open and closed are $\emptyset$ and $C$.
    \begin{proof}
        To prove that the only subsets of $C$ that are both open and closed are $\emptyset$ and $C$, it will suffice to show that if $A\subset C$ is both open and closed, then $A=\emptyset$ or $A=C$. Let $A\subset C$ be both open and closed. We divide into two cases ($A=\emptyset$ and $A\neq\emptyset$). If $A=\emptyset$, then we are done. On the other hand, if $A\neq\emptyset$, we have a bit more work to do. Basically, we will end up proving that the facts that $A$ is open, $A$ is closed, and $A\neq\emptyset$ imply that $A=C$. Let's begin.\par
        First off, the fact that $A$ is closed implies by Definition \ref{dfn:4.8} that $C\setminus A$ is open. Additionally, we have by Script \ref{sct:1} that $A\cap(C\setminus A)=\emptyset$ and $A\cup(C\setminus A)=C$. Now suppose for the sake of contradiction that $A\neq C$. It follows since $A\subset C$ that we must have $C\not\subset A$, i.e., there is some object in $C$ that is not an element of $A$. This object would clearly be an element of $C\setminus A$ in this case, meaning that $C\setminus A$ is nonempty. Thus, we have that $A$ and $C\setminus A$ are disjoint, open, nonempty sets such that $A\cup(C\setminus A)=C$. Consequently, by consecutive applications of Definition \ref{dfn:4.22}, we know that $C$ is disconnected, i.e., $C$ is not connected. But this contradicts Axiom \ref{axm:5.4}, which asserts that $C$ is connected. Therefore, we must have that $A=C$, as desired.
    \end{proof}
\end{theorem}

\begin{theorem}\label{trm:5.2}
    For all $x,y\in C$, if $x<y$, then there exists a point $z\in C$ such that $z$ is between $x$ and $y$.
    \begin{proof}
        Suppose for the sake of contradiction that no point $z\in C$ exists such that $z$ is between $x$ and $y$. To find a contradiction, we will let $A=\{c\in C\mid c<y\}$ and $B=\{c\in C\mid x<c\}$ and prove that $A\cup B=C$, and that $A$ and $B$ are disjoint, nonempty, open sets. This will imply that $C$ is disconnected, contradicting Axiom \ref{axm:5.4}. Let's begin.\par
        Suppose for the sake of contradiction that $C\neq A\cup B$. Then by Theorem \ref{trm:1.7}, $C\not\subset A\cup B$ or $A\cup B\not\subset C$. Since $A\subset C$ and $B\subset C$ by their definitions, we have $A\cup B\subset C$, so it must be that $C\not\subset A\cup B$. Thus, by Definition \ref{dfn:1.3}, there exists a point $p\in C$ such that $p\notin A\cup B$. From the latter condition, we have by Definition \ref{dfn:1.5} that $p\notin A$ and $p\notin B$. It follows from the definitions of $A$ and $B$ that $p\notin C$, or $p\not<y$ and $x\not<p$. But we know that $p\in C$, so it must be that $p\not<y$ and $x\not<p$. Equivalently, $p\geq y$ and $x\geq p$. But this implies that $x\geq y$, which contradicts the fact that $x<y$ by hypothesis. Therefore, we must have $C=A\cup B$, as desired.\par
        Suppose for the sake of contradiction that $A$ and $B$ are not disjoint. Then by Definition \ref{dfn:1.9}, $A\cap B\neq\emptyset$. Thus, Definition \ref{dfn:1.8} tells us that there exists some object $p\in A\cap B$. By Definition \ref{dfn:1.6}, this implies that $p\in A$ and $p\in B$. It follows by the definitions of $A$ and $B$ that $p\in C$, $p<y$, and $x<p$. Since $x<p<y$, Definition \ref{dfn:3.6} tells us that $p$ is between $x$ and $y$, contradicting the supposition that no such point exists. Therefore, $A$ and $B$ are disjoint, as desired.\par
        To prove that $A$ and $B$ are nonempty, Definition \ref{dfn:1.8} tells us that it will suffice to show that there exists an object in each set. Since $x\in C$ and $x<y$, $x\in A$. Similarly, since $y\in C$ and $x<y$, $y\in B$. Therefore, $A$ and $B$ are nonempty, as desired.\par
        By Corollary \ref{cly:4.13}, $A$ and $B$ are open, as desired.\par
        Since $C$ can be written as $A\cup B$ where $A$ and $B$ are disjoint, nonempty, open sets, we have by Definition \ref{dfn:4.22} that $C$ is disconnected. But this contradicts Axiom \ref{axm:5.4}, which asserts that $C$ is connected. Therefore, there must exist a point $z\in C$ such that $z$ is between $x$ and $y$, as desired.
    \end{proof}
\end{theorem}

\begin{corollary}\label{cly:5.3}
    Every region is infinite.
    \begin{proof}
        Let $\underline{ab}$ be a region, and suppose for the sake of contradiction that $\underline{ab}$ is finite. Then by Definitions \ref{dfn:1.30} and \ref{dfn:1.33}, $\underline{ab}=\emptyset$, or $\underline{ab}$ has cardinality $n$. We divide into two cases. Suppose first that $\underline{ab}=\emptyset$. Then by Definitions \ref{dfn:3.10} and \ref{dfn:3.6}, no point $p$ exists such that $a<p<b$. Thus, by the contrapositive of Theorem \ref{trm:5.2}, $a=b$. But this implies by Definition \ref{dfn:3.10} that $\underline{ab}$ is not a region (since $a\not<b$), a contradiction. Now suppose that $\underline{ab}$ has cardinality $n$. Then by Theorem \ref{trm:3.5}, the symbols $a_1,\dots,a_n$ may be assigned to each point of $\underline{ab}$ so that $a_1<a_2<\dots<a_n$. But by Theorem \ref{trm:5.2}, there exists a point $z\in C$ such that $z$ is between $a$ and $a_1$. Since $a<z<a_1<b$, we clearly have that $z\in\underline{ab}$, yet it was not assigned a symbol $a_k$, a contradiction. Therefore, $\underline{ab}$ is infinite, as desired.
    \end{proof}
\end{corollary}

\begin{corollary}\label{cly:5.4}\marginnote{\emph{12/1:}}
    Every point of $C$ is a limit point of $C$.
    \begin{proof}
        Let $p$ be an arbitrary element of $C$. To prove that $p$ is a limit point of $C$, Definition \ref{dfn:3.13} tells us that it will suffice to show that for all regions $R$ with $p\in R$, $R\cap(C\setminus\{p\})\neq\emptyset$. Let $R$ be an arbitrary region with $p\in R$. By Corollary \ref{cly:5.3}, $R$ is infinite, so there exists a point $q\in R$ such that $q\neq p$. Additionally, since $q\in R$, we have $q\in C$. Thus, since $q\in C$ and $q\neq p$ (i.e., $q\notin\{p\}$), we have by Definition \ref{dfn:1.11} that $q\in C\setminus\{p\}$. This combined with the fact that $q\in R$ implies by Definition \ref{dfn:1.6} that $q\in R\cap(C\setminus\{p\})$, so $R\cap(C\setminus\{p\})\neq\emptyset$, as desired.
    \end{proof}
\end{corollary}

\begin{corollary}\label{cly:5.5}
    Every point of the region $\underline{ab}$ is a limit point of $\underline{ab}$.
    \begin{proof}
        Suppose for the sake of contradiction that there exists a point $p\in\underline{ab}$ such that $p\notin LP(\underline{ab})$. Then since $p\notin LP(\underline{ab})$, we have by Definition \ref{dfn:3.13} that there exists a region $R$ with $p\in R$ such that $R\cap(\underline{ab}\setminus\{p\})=\emptyset$. It follows that from the facts that $p\in R$, $p\in\underline{ab}$, and $R\cap(\underline{ab}\setminus\{p\})=\emptyset$ that $R\cap\underline{ab}=\{p\}$. Additionally, since $R$ and $\underline{ab}$ are two regions with a point in common (namely $p$), Theorem \ref{trm:3.18} asserts that $R\cap\underline{ab}$ is a region. Consequently, by Corollary \ref{cly:5.3}, $R\cap\underline{ab}$ is infinite. But this contradicts the result that $R\cap\underline{ab}=\{p\}$, a notably finite set. Therefore, it must be that every point of the region $\underline{ab}$ is a limit point of $\underline{ab}$.
    \end{proof}
\end{corollary}

\begin{definition}\label{dfn:5.6}
    Let $X$ be a subset of $C$. A point $u$ is called an \textbf{upper bound} of $X$ if for all $x\in X$, $x\leq u$. A point $l$ is called a \textbf{lower bound} of $X$ if for all $x\in X$, $l\leq x$. If there exists an upper bound of $X$, then we say that $X$ is \textbf{bounded above}. If there exists a lower bound of $X$, then we say that $X$ is \textbf{bounded below}. If $X$ is bounded above and below, then we simply say that $X$ is \textbf{bounded}.
\end{definition}

\begin{definition}\label{dfn:5.7}
    Let $X$ be a subset of $C$. We say that $u$ is a \textbf{least upper bound} of $X$ and write $u=\sup X$ if:
    \begin{enumerate}
        \item $u$ is an upper bound of $X$;
        \item if $u'$ is an upper bound of $X$, then $u\leq u'$.
    \end{enumerate}
    We say that $l$ is a \textbf{greatest lower bound} and write $l=\inf X$ if:
    \begin{enumerate}
        \item $l$ is a lower bound of $X$;
        \item if $l'$ is a lower bound of $X$, then $l'\leq l$.
    \end{enumerate}
    The notation $\sup$ comes from the word \textbf{supremum}, which is another name for least upper bound. The notation $\inf$ comes from the word \textbf{infimum}, which is another name for greatest lower bound.
\end{definition}

\begin{exercise}\label{exr:5.8}
    If $\sup X$ exists, then it is unique, and similarly for $\inf X$.
    \begin{proof}
        Let $X$ be a subset of a continuum $C$ such that $\sup X$ exists, and suppose that both $u$ and $u'$ are least upper bounds of $X$. It follows from the supposition and Definition \ref{dfn:5.7} that $u,u'$ are both upper bounds of $X$. Thus, since $u$ is a least upper bound of $X$ and $u'$ is an upper bound of $X$, we have by Definition \ref{dfn:5.7} again that $u\leq u'$. By a symmetric argument, we also have that $u'\leq u$. But since $u\leq u'$ and $u'\leq u$, $u=u'$, proving the uniqueness of $\sup X$.\par
        The proof is symmetric for $\inf X$.
    \end{proof}
\end{exercise}

\begin{exercise}\label{exr:5.9}
    If $X$ has a first point $L$, then $\inf X$ exists and equals $L$. Similarly, if $X$ has a last point $U$, then $\sup X$ exists and equals $U$.
    \begin{proof}
        Let $L$ be the first point of $X$. Then by Definition \ref{dfn:3.3}, for all $x\in X$, $L\leq x$. Thus, by Definition \ref{dfn:5.6}, $L$ is a lower bound of $X$. Now suppose for the sake of contradiction that there exists a lower bound $L'$ of $X$ such that $L'>L$. Since $L'$ is a lower bound, Definition \ref{dfn:5.6} implies that for all $x\in X$, $L'\leq x$. But $L$ is an element of $X$ and $L<L'$, a contradiction. Therefore, if $L'$ is a lower bound of $X$, then $L'\leq L$. This result coupled with the fact that $L$ is a lower bound of $X$ implies by Definition \ref{dfn:5.7} that $L=\inf X$.\par
        The proof is symmetric in the other case.
    \end{proof}
\end{exercise}

\begin{exercise}\label{exr:5.10}
    For this exercise, we assume that $C=\R$. Find $\sup X$ and $\inf X$ for each of the following subsets of $\R$, or state that they do not exist. You need not give proofs.
    \begin{enumerate}
        \item $X=\N$.
        \begin{proof}[Answer]
            $\sup X$ does not exist because the natural numbers continue on forever to positive infinity. However, $\inf X=1$ since we know that $1\leq n$ for all $n\in\N$.
        \end{proof}
        \item $X=\Q$.
        \begin{proof}[Answer]
            Neither $\sup X$ nor $\inf X$ exists because the rational numbers continue on forever to both positive and negative infinity.
        \end{proof}
        \item $X=\left\{ \frac{1}{n}\mid n\in\N \right\}$.
        \begin{proof}[Answer]
            For $n=1$, $\frac{1}{n}=1$. From here, as $n$ increases, $\frac{1}{n}$ decreases asymptotically toward zero but always remains a positive nonzero rational number. Thus, $\sup X=1$ and $\inf X=0$.
        \end{proof}
        \item $X=\{x\in\R\mid 0<x<1\}$.
        \begin{proof}[Answer]
            $\sup X=1$ and $\inf X=0$. In the case of $\sup X$, any number slightly less than 1 would be included in $X$ and have a number in $X$ between it and 1 by Theorem \ref{trm:5.2}, i.e., greater than it. A symmetric argument can treat the other case.
        \end{proof}
        \item $X=\{3\}\cup\{x\in\R\mid-7\leq x\leq -5\}$.
        \begin{proof}[Answer]
            $\sup X=3$ (3 is the greatest element of the set) and $\inf X=-7$ (for a similar reason to part 4, above).
        \end{proof}
    \end{enumerate}
\end{exercise}

\begin{lemma}\label{lem:5.11}
    Suppose that $X\subset C$ and $s=\sup X$. If $p<s$, then there exists an $x\in X$ such that $p<x\leq s$. Similarly, suppose that $X\subset C$ and $l=\inf X$. If $l<p$, then there exists an $x\in X$ such that $l\leq x<p$.
    \begin{proof}
        Suppose for the sake of contradiction that for some $p<s$, no $x\in X$ exists such that $p<x\leq s$. Since $s$ is a least upper bound of $X$, Definitions \ref{dfn:5.7} and \ref{dfn:5.6} imply\footnote{Technically, Definition \ref{dfn:5.7} implies that $s$ is an upper bound of $X$ and Definition \ref{dfn:5.6} implies based off of this result that for all $x\in X$, $x\leq s$. However, to avoid having to write this every time, I will shorthand this concept in this fashion.} that for all $x\in X$, $x\leq s$. Consequently, by the supposition, it is true that for all $x\in X$, $x\leq p$ (if there existed an $x>p$, then this point would satisfy $p<x\leq s$, contradicting the supposition). Thus, by Definition \ref{dfn:5.6}, $p$ is a upper bound of $X$. But since $p<s$, it is not true that $s\leq s'$ for all upper bounds $s'$ of $X$, meaning by Definition \ref{dfn:5.7} that $s$ is not a least upper bound of $X$, a contradiction. Therefore, if $p<s$, then there exists an $x\in X$ such that $p<x\leq s$, as desired.\par
        The proof is symmetric in the other case.
    \end{proof}
\end{lemma}

\begin{theorem}\label{trm:5.12}
    Let $a<b$. The least upper bound and greatest lower bound of the region $\underline{ab}$ are $\sup\underline{ab}=b$ and $\inf\underline{ab}=a$.
    \begin{proof}
        To prove that $\sup\underline{ab}=b$, Definition \ref{dfn:5.7} tells us that it will suffice to show that $b$ is an upper bound of $\underline{ab}$ and that if $u$ is an upper bound of $\underline{ab}$, then $b\leq u$. For the first condition, Definition \ref{dfn:5.6} tells us that it will suffice to confirm that for all $x\in\underline{ab}$, $x\leq b$. Let $x$ be an arbitrary element of $\underline{ab}$. Then by Definitions \ref{dfn:3.10} and \ref{dfn:3.6}, we know that $a<x<b$, i.e., $x\leq b$, as desired. For the second condition, suppose for the sake of contradiction that $u$ is an upper bound of $\underline{ab}$ such that $u<b$. Then by Definition \ref{dfn:5.6}, for all $x\in\underline{ab}$, $x\leq u$. Additionally, since $\underline{ab}$ is infinite by Corollary \ref{cly:5.3}, we know that at least one such $x$ exists, which we shall hereafter refer to as $y$. Note that as an element of $\underline{ab}$, $y$ satisfies $a<y<b$ by Definitions \ref{dfn:3.10} and \ref{dfn:3.6}. Furthermore, since $u<b$, Theorem \ref{trm:5.2} implies that there exists a point $z$ such that $z$ is between $u$ and $b$. Thus, by Definition \ref{dfn:3.6}, $u<z<b$. Combining the last few results, we have $a<y\leq u<z<b$. Consequently, since $a<z<b$, we have by Definitions \ref{dfn:3.6} and \ref{dfn:3.10} that $z\in\underline{ab}$ and $u<z$, contradicting the statement that for all $x\in\underline{ab}$, $x\leq u$. Therefore, if $u$ is an upper bound of $\underline{ab}$, then $b\leq u$, as desired.\par
        The proof is symmetric in the other case.
    \end{proof}
\end{theorem}

\begin{lemma}\label{lem:5.13}\marginnote{\emph{12/3:}}
    Let $X$ be a subset of $C$. Suppose that $\sup X$ exists and $\sup X\notin X$. Then $\sup X$ is a limit point of $X$. The same holds for $\inf X$.
    \begin{proof}
        To prove that $\sup X$ is a limit point of $X$, Definition \ref{dfn:3.13} tells us that it will suffice to verify that for all regions $\underline{ab}$ with $\sup X\in\underline{ab}$, we have $\underline{ab}\cap(X\setminus\{\sup X\})\neq\emptyset$. Let $\underline{ab}$ be an arbitrary region with $\sup X\in\underline{ab}$. Then by Definitions \ref{dfn:3.10} and \ref{dfn:3.6}, $a<\sup X<b$. It follows by Lemma \ref{lem:5.11} that there exists an $x\in X$ such that $a<x\leq\sup X$. Additionally, since $x\in X$ and $\sup X\notin X$, we cannot have $\sup X=x$, meaning that $a<x<\sup X$. Combining the last few results, we have $a<x<\sup X<b$. Thus, by Definitions \ref{dfn:3.6} and \ref{dfn:3.10}, $x\in\underline{ab}$. Consequently, since $x\in X$ and $x\neq\sup X$ implies $x\notin\{\sup X\}$, Definition \ref{dfn:1.11} asserts that $x\in X\setminus\{\sup X\}$. Therefore, since we also know that $x\in\underline{ab}$, we have by Definition \ref{dfn:1.6} that $x\in\underline{ab}\cap(X\setminus\{\sup X\})$, meaning by Definition \ref{dfn:1.8} that $\underline{ab}\cap(X\setminus\{\sup X\})\neq\emptyset$, as desired.\par
        The proof is symmetric in the other case.
    \end{proof}
\end{lemma}

\begin{corollary}\label{cly:5.14}
    Both $a$ and $b$ are limit points of the region $\underline{ab}$.
    \begin{proof}
        Clearly, $\underline{ab}\subset C$. Additionally, by Theorem \ref{trm:5.12}, $\sup\underline{ab}$ and $\inf\underline{ab}$ exist and are equal to $b$ and $a$, respectively. Furthermore, it follows from Definition \ref{dfn:3.10} that neither $b$ nor $a$ (i.e., neither $\sup\underline{ab}$ nor $\inf\underline{ab}$) are elements of $\underline{ab}$. Therefore, by Lemma \ref{lem:5.13}, $\inf\underline{ab}=a$ and $\sup\underline{ab}=b$ are limit points of the region $\underline{ab}$.
    \end{proof}
\end{corollary}

\begin{corollary}\label{cly:5.15}
    Let $[a,b]$ denote the closure $\overline{\underline{ab}}$ of the region $\underline{ab}$. Then $[a,b]=\{x\in C\mid a\leq x\leq b\}$.
    \begin{proof}
        To prove that $[a,b]=\{x\in C\mid a\leq x\leq b\}$, Definition \ref{dfn:1.2} tells us that it will suffice to show that every element $y\in[a,b]$ is an element of $\{x\in C\mid a\leq x\leq b\}$ and vice versa.\par
        First, let $y\in[a,b]$. Then by Definition \ref{dfn:4.4}, $y\in\underline{ab}\cup LP(\underline{ab})$. It follows by Definition \ref{dfn:1.5} that $y\in\underline{ab}$ or $y\in LP(\underline{ab})$. We divide into two cases. Suppose first that $y\in\underline{ab}$. Then by Definitions \ref{dfn:3.10} and \ref{dfn:3.6}, $y\in C$ and $a<y<b$. The latter condition implies that $a\leq y\leq b$. Therefore, since $y\in C$ and $a\leq y\leq b$, we have that $y\in\{x\in C\mid a\leq x\leq b\}$ in this case. Now suppose that $y\in LP(\underline{ab})$. Then by Lemma \ref{lem:3.17}, $y\notin\ext\underline{ab}$. By Definition \ref{dfn:3.15}, this implies that $y\notin C\setminus(\{a\}\cup\underline{ab}\cup\{b\})$. It follows by Definition \ref{dfn:1.11} that $y\notin C$ or $y\in\{a\}\cup\underline{ab}\cup\{b\}$. Now as a limit point of $\underline{ab}$, Definition \ref{dfn:3.13} asserts that $y\in C$, so it must be that $y\in\{a\}\cup\underline{ab}\cup\{b\}$. Consequently, by two applications of Definition \ref{dfn:1.5} as well as Definitions \ref{dfn:3.10} and \ref{dfn:3.6}, $y=a$, $a<y<b$, or $y=b$. In other words, $a\leq y\leq b$. Therefore, since $y\in C$ and $a\leq y\leq b$, we have that $y\in\{x\in C\mid a\leq x\leq b\}$ in this case, too, as desired.\par
        Now let $y\in\{x\in C\mid a\leq x\leq b\}$. Then $y\in C$ and $a\leq y\leq b$. We divide into three cases ($y=a$, $y=b$, and $a<y<b$). Suppose first that $y=a$. Then by Corollary \ref{cly:5.14}, $y\in LP(\underline{ab})$. Thus, by Definition \ref{dfn:1.5}, $y\in\underline{ab}\cup LP(\underline{ab})$. Consequently, by Definition \ref{dfn:4.4}, $y\in[a,b]$ in this case. The proof of the second case is symmetric to that of the first. Lastly, suppose that $a<y<b$. Then by Definitions \ref{dfn:3.6} and \ref{dfn:3.10}, $y\in\underline{ab}$. Thus, by Definition \ref{dfn:1.5}, $y\in\underline{ab}\cup LP(\underline{ab})$. Consequently, by Definition \ref{dfn:4.4}, $y\in[a,b]$ in this case, too, as desired.
    \end{proof}
\end{corollary}

\begin{lemma}\label{lem:5.16}
    Let $X\subset C$ and define:
    \begin{align*}
        \Psi(X) &= \{x\in C\mid x\text{ is not an upper bound of }X\}&
        \Omega(X) &= \{x\in C\mid x\text{ is not a lower bound of }X\}
    \end{align*}
    Then both $\Psi(X)$ and $\Omega(X)$ are open.
    \begin{proof}
        We will take this one set at a time.\par
        To prove that $\Psi(X)$ is open, Theorem \ref{trm:4.10} tells us that it will suffice to confirm that for all $y\in\Psi(X)$, there exists a region containing $y$ that is a subset of $\Psi(X)$. Let $y$ be an arbitrary element of $\Psi(X)$. Then by the definition of $\Psi(X)$, $y$ is not an upper bound of $X$. Thus, by Definition \ref{dfn:5.6}, there exists some $x\in X$ such that $x>y$. Now let $a\in C$ be a point such that $a<y$ (Axiom \ref{axm:3.3} and Definition \ref{dfn:3.3} imply the existence of such a point) and consider the region $\underline{ax}$. We will demonstrate that $\underline{ax}$ is the desired region, i.e., that $y\in\underline{ax}$ and $\underline{ax}\subset\Psi(X)$. For the first condition, since $a<y<x$, it immediately follows from Definitions \ref{dfn:3.6} and \ref{dfn:3.10} that $y\in\underline{ax}$, as desired. As to the second condition, Definition \ref{dfn:1.3} tells us that it will suffice to show that every element $z\in\underline{ax}$ is an element of $\Psi(X)$. Let $z$ be an arbitrary element of $\underline{ax}$. Then by Definitions \ref{dfn:3.10} and \ref{dfn:3.6}, $z<x$. Since $z$ is less than an element of $X$, Definition \ref{dfn:5.6} asserts that $z$ is not an upper bound of $X$. Thus, by the definition of $\Psi(X)$, $z\in\Psi(X)$, as desired. Therefore, for all $y\in\Psi(X)$, there exists a region containing $y$ that is a subset of $\Psi(X)$.\par
        The proof is symmetric in the other case.
    \end{proof}
\end{lemma}




\end{document}