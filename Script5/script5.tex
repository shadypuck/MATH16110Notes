\documentclass[../main.tex]{subfiles}

\pagestyle{main}
\renewcommand{\chaptermark}[1]{\markboth{\chaptername\ \thechapter}{#1}}
\setcounter{chapter}{4}
\setcounter{axiom}{3}

\begin{document}




\chapter{Connectedness and Boundedness}\label{sct:5}
\section{Journal}
\begin{axiom}\label{axm:5.4}\marginnote{\emph{11/19:}}
    A continuum is connected.
\end{axiom}

\begin{theorem}\label{trm:5.1}
    The only subsets of a continuum $C$ that are both open and closed are $\emptyset$ and $C$.
    \begin{proof}
        To prove that the only subsets of $C$ that are both open and closed are $\emptyset$ and $C$, it will suffice to show that if $A\subset C$ is both open and closed, then $A=\emptyset$ or $A=C$. Let $A\subset C$ be both open and closed. We divide into two cases ($A=\emptyset$ and $A\neq\emptyset$). If $A=\emptyset$, then we are done. On the other hand, if $A\neq\emptyset$, we have a bit more work to do. Basically, we will end up proving that the facts that $A$ is open, $A$ is closed, and $A\neq\emptyset$ imply that $A=C$. Let's begin.\par
        First off, the fact that $A$ is closed implies by Definition \ref{dfn:4.8} that $C\setminus A$ is open. Additionally, we have by Script \ref{sct:1} that $A\cap(C\setminus A)=\emptyset$ and $A\cup(C\setminus A)=C$. Now suppose for the sake of contradiction that $A\neq C$. It follows since $A\subset C$ that we must have $C\not\subset A$, i.e., there is some object in $C$ that is not an element of $A$. This object would clearly be an element of $C\setminus A$ in this case, meaning that $C\setminus A$ is nonempty. Thus, we have that $A$ and $C\setminus A$ are disjoint, open, nonempty sets such that $A\cup(C\setminus A)=C$. Consequently, by consecutive applications of Definition \ref{dfn:4.22}, we know that $C$ is disconnected, i.e., $C$ is not connected. But this contradicts Axiom \ref{axm:5.4}, which asserts that $C$ is connected. Therefore, we must have that $A=C$, as desired.
    \end{proof}
\end{theorem}

\begin{theorem}\label{trm:5.2}
    For all $x,y\in C$, if $x<y$, then there exists a point $z\in C$ such that $z$ is between $x$ and $y$.
    \begin{proof}
        Suppose for the sake of contradiction that no point $z\in C$ exists such that $z$ is between $x$ and $y$. To find a contradiction, we will let $A=\{c\in C\mid c<y\}$ and $B=\{c\in C\mid x<c\}$ and prove that $A\cup B=C$, and that $A$ and $B$ are disjoint, nonempty, open sets. This will imply that $C$ is disconnected, contradicting Axiom \ref{axm:5.4}. Let's begin.\par
        Suppose for the sake of contradiction that $C\neq A\cup B$. Then by Theorem \ref{trm:1.7}, $C\not\subset A\cup B$ or $A\cup B\not\subset C$. Since $A\subset C$ and $B\subset C$ by their definitions, we have $A\cup B\subset C$, so it must be that $C\not\subset A\cup B$. Thus, by Definition \ref{dfn:1.3}, there exists a point $p\in C$ such that $p\notin A\cup B$. From the latter condition, we have by Definition \ref{dfn:1.5} that $p\notin A$ and $p\notin B$. It follows from the definitions of $A$ and $B$ that $p\notin C$, or $p\not<y$ and $x\not<p$. But we know that $p\in C$, so it must be that $p\not<y$ and $x\not<p$. Equivalently, $p\geq y$ and $x\geq p$. But this implies that $x\geq y$, which contradicts the fact that $x<y$ by hypothesis. Therefore, we must have $C=A\cup B$, as desired.\par
        Suppose for the sake of contradiction that $A$ and $B$ are not disjoint. Then by Definition \ref{dfn:1.9}, $A\cap B\neq\emptyset$. Thus, Definition \ref{dfn:1.8} tells us that there exists some object $p\in A\cap B$. By Definition \ref{dfn:1.6}, this implies that $p\in A$ and $p\in B$. It follows by the definitions of $A$ and $B$ that $p\in C$, $p<y$, and $x<p$. Since $x<p<y$, Definition \ref{dfn:3.6} tells us that $p$ is between $x$ and $y$, contradicting the supposition that no such point exists. Therefore, $A$ and $B$ are disjoint, as desired.\par
        To prove that $A$ and $B$ are nonempty, Definition \ref{dfn:1.8} tells us that it will suffice to show that there exists an object in each set. Since $x\in C$ and $x<y$, $x\in A$. Similarly, since $y\in C$ and $x<y$, $y\in B$. Therefore, $A$ and $B$ are nonempty, as desired.\par
        By Corollary \ref{cly:4.13}, $A$ and $B$ are open, as desired.\par
        Since $C$ can be written as $A\cup B$ where $A$ and $B$ are disjoint, nonempty, open sets, we have by Definition \ref{dfn:4.22} that $C$ is disconnected. But this contradicts Axiom \ref{axm:5.4}, which asserts that $C$ is connected. Therefore, there must exist a point $z\in C$ such that $z$ is between $x$ and $y$, as desired.
    \end{proof}
\end{theorem}

\begin{corollary}\label{cly:5.3}
    Every region is infinite.
    \begin{proof}
        Let $\underline{ab}$ be a region, and suppose for the sake of contradiction that $\underline{ab}$ is finite. Then by Definitions \ref{dfn:1.30} and \ref{dfn:1.33}, $\underline{ab}=\emptyset$, or $\underline{ab}$ has cardinality $n$. We divide into two cases. Suppose first that $\underline{ab}=\emptyset$. Then by Definitions \ref{dfn:3.10} and \ref{dfn:3.6}, no point $p$ exists such that $a<p<b$. Thus, by the contrapositive of Theorem \ref{trm:5.2}, $a=b$. But this implies by Definition \ref{dfn:3.10} that $\underline{ab}$ is not a region (since $a\not<b$), a contradiction. Now suppose that $\underline{ab}$ has cardinality $n$. Then by Theorem \ref{trm:3.5}, the symbols $a_1,\dots,a_n$ may be assigned to each point of $\underline{ab}$ so that $a_1<a_2<\dots<a_n$. But by Theorem \ref{trm:5.2}, there exists a point $z\in C$ such that $z$ is between $a$ and $a_1$. Since $a<z<a_1<b$, we clearly have that $z\in\underline{ab}$, yet it was not assigned a symbol $a_k$, a contradiction. Therefore, $\underline{ab}$ is infinite, as desired.
    \end{proof}
\end{corollary}




\end{document}