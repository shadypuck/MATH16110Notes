\documentclass[../main.tex]{subfiles}

\pagestyle{main}
\renewcommand{\chaptermark}[1]{\markboth{\chaptername\ \thechapter}{#1}}
\setcounter{chapter}{2}
\externaldocument{../main}

\begin{document}




\chapter{Introducing a Continuum}
\section{Journal}
\begin{axiom}\label{axm:3.1}\marginnote{\emph{10/20:}}
    A continuum is a nonempty set $C$.
\end{axiom}

\begin{definition}\label{dfn:3.1}
    Let $X$ be a set. An \textbf{ordering} on the set $X$ is a subset $<$ of $X\times X$ with elements $(x,y)\in\, <$ written as $x<y$, satisfying the following properties:
    \begin{enumerate}[label={\alph*)},ref={\thetheorem\alph*}]
        \item \label{dfn:3.1a}(\emph{Trichotomy}) For all $x,y\in X$, exactly one of the following holds: $x<y$, $y<x$, or $x=y$.
        \item \label{dfn:3.1b}(\emph{Transitivity}) For all $x,y,z\in X$, if $x<y$ and $y<z$, then $x<z$.
    \end{enumerate}
\end{definition}

\begin{remark}\label{rmk:3.2}\leavevmode
    \begin{enumerate}[label={\alph*)}]
        \item In mathematics, "or" is understood to be inclusive unless stated otherwise. So in Definition \ref{dfn:3.1a} above, the word "exactly" is needed.
        \item $x<y$ may also be written as $y>x$.
        \item By $x\leq y$, we mean $x<y$ or $x=y$; similarly for $x\geq y$.
        \item We often refer to elements of a continuum $C$ as \textbf{points}.
    \end{enumerate}
\end{remark}

\begin{axiom}\label{axm:3.2}
    A continuum $C$ has an ordering $<$.
\end{axiom}

\begin{definition}\label{dfn:3.3}
    If $A\subset C$, then a point $a\in A$ is a \textbf{first} point of $A$ if for every element $x\in A$, either $a<x$ or $a=x$. Similarly, a point $b\in A$ is called a \textbf{last} point of $A$ if, for every $x\in A$, either $x<b$ or $x=b$.
\end{definition}

\begin{lemma}\label{lem:3.4}
    If $A$ is a nonempty, finite subset of a continuum $C$, then $A$ has a first and last point.
    \begin{lemma*}
        Let $A$ be a nonempty, finite set \textup{(}i.e., $|A|=n$ for some $n\in\N$\textup{)}, let $a$ be any element of $A$, and let the set $B=A\setminus\{a\}$. Then $|B|=n-1$.
        \begin{proof}
            We first prove that $|\{a\}|=1$. By Definition \ref{dfn:1.33}, to do so, it will suffice to find a bijection $f:\{a\}\to[1]$. Since $[1]=\{1\}$ by Definition \ref{dfn:1.29}, $f:\{a\}\to\{1\}$ defined by $f(a)=1$ is clearly such a bijection. We now demonstrate that $B\cap\{a\}=(A\setminus\{a\})\cap\{a\}=\emptyset$. The previous two results combined with the fact that $B\cup\{a\}=(A\setminus\{a\})\cup\{a\}=A$ imply by Theorem \ref{trm:1.34b} that $|A|=|B|+|\{a\}|$. It follows that $n=|B|+1$, so $|B|=n-1$.
        \end{proof}
    \end{lemma*}
    \begin{proof}[Proof of Lemma \ref{lem:3.4}]
        We consider first points herein (the proof is symmetric for last points). If $A$ is a finite set, then by Definition \ref{dfn:1.30}, $|A|=n$ for some $n\in\N$. Thus, if we prove the claim for each $n\in\N$ individually, we will have proven the claim. To prove a property pertaining to any natural number, we induct on $n$.\par
        For the base case $n=1$, there is only one element (which we may call $a$) in $A$. Since $a=a$, i.e., "for every $x\in A$, either $a<x$ or $a=x$" is a true statement, it follows by Definition \ref{dfn:3.3} that $A$ has a first point. Now suppose inductively that we have proven the claim for $n$, i.e., we know that if $A$ is a nonempty, finite subset of a continuum $C$ with $|A|=n$, then $A$ has a first point. We wish to prove the same claim if $|A|=n+1$. Let $a$ be an arbitrary element of $A$, and consider the set $B=A\setminus\{a\}$. By the lemma, $|B|=n$. Consequently, the induction hypothesis applies and asserts that $B$ has a first point $a_0$. Clearly, $a_0$ is also an element of $A$, but it may or may not be the first point of $A$ (the first point may now be $a$). Since $C$ has an ordering $<$ (see Axiom \ref{axm:3.2}), Definition \ref{dfn:3.1} asserts that either $a<a_0$, $a_0<a$, or $a=a_0$. We now divide into three cases. If $a<a_0$, then since $a_0\leq x$ for all $x\in A$ by Definition \ref{dfn:3.3}, Definition \ref{dfn:3.1} implies that $a\leq x$ for all $x\in A$. Thus, by Definition \ref{dfn:3.3}, $a$ is the first point in $A$, and we have proven the claim for $|A|=n+1$ in this case. If $a_0<a$, then it is still true that $a_0\leq x$ for all $x\in A$. This means by Definition \ref{dfn:3.3} that $a_0$ is still the first point in $A$, proving the claim for $|A|=n+1$ in this case. If $a=a_0$, then $a\in B$, contradicting the fact that $B=A\setminus\{a\}$, so we need not consider this final case. This closes the induction.
    \end{proof}
\end{lemma}

\begin{theorem}\label{trm:3.5}
    Suppose that $A$ is a set of $n$ distinct points in a continuum $C$, or in other words, $A\subset C$ has cardinality $n$. Then the symbols $a_1,\dots,a_n$ may be assigned to each point of $A$ so that $a_1<a_2<\cdots<a_n$, i.e., $a_i<a_{i+1}$ for all $1\leq i\leq n-1$.
    \begin{proof}
        We divide into two cases ($|A|=0$ and $|A|\in\N$).\par
        If $|A|=0$, then the statements "the symbols $a_1,\dots,a_n$ may be assigned to each point of $A$" and "$a_i<a_{i+1}$ for all $1\leq i\leq n-1=-1$" are both vacuously true.\par
        If $|A|\in\N$, we induct on $|A|=n$. For the base case $n=1$, denote the single element of $A$ by $a_1$. Since $a_i<a_{i+1}$ for all $1\leq i\leq n-1=0$ is vacuously true, the base case holds. Now suppose inductively that we have proven the claim for $n$, i.e., for a set $A\subset C$ satisfying $|A|=n$, the symbols $a_1,\dots,a_n$ may be assigned to each point of $A$ so that $a_1<a_2<\cdots<a_n$. We now wish to prove the claim with regard to a set $A\subset C$ with $|A|=n+1$. By Lemma \ref{lem:3.4}, there is a last point $a_{n+1}\in A$, which may be denoted as such (we will rigorously confirm this later). Since the set $A\setminus\{a_{n+1}\}$ has cardinality $n$ (see the lemma from Lemma \ref{lem:3.4}), we have by the induction hypothesis that its $n$ elements can be named $a_1,\dots,a_n$ and ordered $a_1<a_2<\cdots<a_n$. Clearly these $n$ elements are elements of $A$ and all that's left to do is determine where $a_{n+1}$ fits into the established order. But by Definition \ref{dfn:3.3}, $x\leq a_{n+1}$ for all $x\in A$, i.e., $x<a_{n+1}$ for all $x\in A\setminus\{a_{n+1}\}$. Consequently, as its name would suggest, it is true that $a_1<a_2<\cdots<a_n<a_{n+1}$, as desired.
    \end{proof}
\end{theorem}

\begin{definition}\label{dfn:3.6}
    If $x,y,z\in C$ and either (i) both $x<y$ and $y<z$ or (ii) both $z<y$ and $y<x$, then we say that $y$ is \textbf{between} $x$ and $z$.
\end{definition}

\begin{corollary}\label{cly:3.7}
    Of three distinct points in a continuum, one must be between the other two.
    \begin{proof}
        Let $A$ be a subset of the described continuum containing the three distinct points. It follows by Theorem \ref{trm:3.5} that the symbols $a_1,a_2,a_3$ may be assigned to each point of $A$ so that $a_1<a_2<a_3$. Thus, $a_1<a_2$ and $a_2<a_3$, so $a_2$ is between $a_1$ and $a_3$ by Definition \ref{dfn:3.6}.
    \end{proof}
\end{corollary}

\begin{axiom}\label{axm:3.3}\marginnote{\emph{10/22:}}
    A continuum $C$ has no first or last point.
\end{axiom}

\begin{definition}\label{dfn:3.8}
    We define an ordering on $\Z$ by $m<n$ if $n=m+c$ for some $c\in\N$.
\end{definition}

\begin{exercise}\label{exr:3.9}\leavevmode
    \begin{enumerate}[label={\alph*)},ref={\thetheorem\alph*}]
        \item \label{exr:3.9a}Prove that with this ordering $\Z$ satisfies Axioms 1-3.
        \begin{proof}
            Clearly, $\Z$ is a nonempty set, so Axiom \ref{axm:3.1} is immediately satisfied.\par\medskip
            Axiom \ref{axm:3.2} asserts that $\Z$ must have an ordering $<$. As such, it will suffice to verify that the ordering given by Definition \ref{dfn:3.8} satisfies the stipulations of Definition \ref{dfn:3.1}. To prove that $<$ satisfies the trichotomy, it will suffice to show that for all $x,y\in\Z$, exactly one of the following holds: $x<y$, $y<x$, or $x=y$.\par
            We first show that \emph{no more than one} of the three statements can simultaneously be true. Let $x,y$ be arbitrary elements of $\Z$. We divide into three cases. First, suppose for the sake of contradiction that $x<y$ and $y<x$. By Definition \ref{dfn:3.8}, this implies that $y=x+c$ and $x=y+c'$ for some $c,c'\in\N$. Substituting, we have $y=y+c'+c$, or $0=c'+c$ by the cancellation law of addition. But since $c',c\in\N$, the closure of addition on $\N$ implies that $(c'+c)\in\N$. Therefore, $c'+c\neq 0$, a contradiction. Second, suppose for the sake of contradiction that $x<y$ and $x=y$. By Definition \ref{dfn:3.8}, this implies that $y=x+c$ for some $c\in\N$. Substituting, we have $y=y+c$, or $0=c$ by the cancellation law of addition. But since $c\in\N$, $c\neq 0$, a contradiction. The proof of the third case ($y<x$ and $x=y$) is symmetric to that of the second case.\par
            We now show that \emph{at least one} of the three statements is always true. Let $x,y$ be arbitrary elements of $\Z$, and suppose for the sake of contradiction that $x\not<y$, $y\not<x$, and $x\neq y$. Since $x\not<y$, $y\neq x+c$ for any $c\in\N$. Equivalently, $y-x\neq c$ for any $c\in\N$, i.e., $(y-x)\notin\N$. Similarly, since $y\not<x$, $x-y\neq c'$ for any $c'\in\N$. Equivalently, $y-x\neq c'$ for any $c'\in -\N$, i.e., $(y-x)\notin -\N$. Lastly, since $x\neq y$, $y-x\neq 0$, i.e., $(y-x)\notin\{0\}$. Since $(y-x)\notin-\N$, $(y-x)\notin\{0\}$, and $(y-x)\notin\N$, Definition \ref{dfn:1.5} implies that $(y-x)\notin(-\N)\cup\{0\}\cup\N$. Consequently, by Script \ref{sct:0}, $(y-x)\notin\Z$. But by the closure of integer subtraction, $(y-x)\in\Z$, a contradiction.\par\smallskip
            To prove that $<$ is transitive, it will suffice to show that for all $x,y,z\in\Z$, if $x<y$ and $y<z$, then $x<z$. Let $x,y,z$ be arbitrary elements of $\Z$ for which it is true that $x<y$ and $y<z$. By Definition \ref{dfn:3.8}, we have $y=x+c$ and $z=y+c'$ for some $c,c'\in\N$. Substituting, we have $z=x+c+c'$. Since $(c+c')\in\N$ by the closure of addition on $\N$, Definition \ref{dfn:3.8} implies that $x<z$.\par\medskip
            Axiom \ref{axm:3.3} asserts that $\Z$ must have no first or last point. Suppose for the sake of contradiction that $\Z$ has some first point $a$. Then by Definition \ref{dfn:3.3}, $a\leq x$ for every $x\in\Z$. However, under the closure of subtraction on $\Z$, $(a-1)\in\Z$. Since $(a-1)+1=a$, Definition \ref{dfn:3.8} asserts that $a-1<a$, a contradiction. The proof is symmetric for the last point.
        \end{proof}
        \item \label{exr:3.9b}Show that for any $p=\eqclass{a}{b}\in\Q$, there is some $(a_1,b_1)\in p$ with $0<b_1$.
        \begin{proof}
            Let $\eqclass{a}{b}$ be an arbitrary element of $\Q$. It follows by Definition \ref{dfn:2.5} that $(a,b)\in X$. Since we also have $(a,b)\sim(a,b)$ by Exercise \ref{exr:2.2e}, Definition \ref{dfn:2.5} implies that $(a,b)\in\eqclass{a}{b}$. By the trichotomy on $\Z$ (see Exercise \ref{exr:3.9a}), we have $0<b$, $b<0$, or $0=b$. We divide into three cases. First, suppose that $0<b$. Then $(a,b)$ is an element $(a_1,b_1)\in\eqclass{a}{b}$ for which $0<b_1$, and we are done. Second, suppose that $b<0$. Since $(-a)(b)=(-b)(a)$, we have by the definition of $\sim$ that $(-a,-b)\sim(a,b)$. Additionally, we have by the closure of integer multiplication that $-a,-b\in\Z$, and since $b\neq 0$ by Exercise \ref{exr:2.2e} and clearly $-1\neq 0$, $-b\neq 0$ by the contrapositive of the zero-product property. Thus, by Exercise \ref{exr:2.2e}, $(-a,-b)\in X$. This coupled with the previously proven fact that $(-a,-b)\sim(a,b)$ implies by Definition \ref{dfn:2.5} that $(-a,-b)\in\eqclass{a}{b}$. Now recall that $b<0$ by hypothesis, so we may use Definition \ref{dfn:3.8} to see that $b+c=0$ for some $c\in\N$. It follows that $-(b+c)=0$, i.e., $-b-c=0$, i.e., $-b=0+c$, meaning that $0<-b$ by Definition \ref{dfn:3.8}. Thus, $(-a,-b)$ is an element $(a_1,b_1)\in\eqclass{a}{b}$ for which $0<b_1$. Third, suppose that $b=0$. But this contradicts Exercise \ref{exr:2.2e} which asserts that $b\neq 0$, so we need not consider this case.
        \end{proof}
        \item \label{exr:3.9c}Define an ordering $\lQ$ on $\Q$ as follows. For $p,q\in\Q$, let $(a_1,b_1)\in p$ be such that $0<b_1$ and let $(a_2,b_2)\in q$ be such that $0<b_2$. Then we define $p\lQ q$ if $a_1b_2<a_2b_1$. Show that $\lQ$ is a well-defined relation on $\Q$.
        \begin{proof}
            For the relation $\lQ$ to be well-defined, Definition \ref{dfn:3.1} tells us that it must satisfy the trichotomy and be transitive.\par\smallskip
            To prove that $\lQ$ satisfies the trichotomy, it will suffice to show that for all $p,q\in\Q$, exactly one of the following holds: $p\lQ q$, $q\lQ p$, or $p=q$.\par
            We first show that \emph{no more than one} of the three statements can be simultaneously true. Let $p,q$ be arbitrary elements of $\Q$, let $(a,b)\in p$ be such that $0<b$ (we know that such an element exists by Exercise \ref{exr:3.9b}\footnote{This justification will not be supplied every subsequent time we choose such an element to make the proof less repetitive.}), and let $(c,d)\in q$ be such that $0<d$. We divide into three cases. First, suppose for the sake of contradiction that $p\lQ q$ and $q\lQ p$. Then $ad<bc$ and $cb<da$ by the definition of $\lQ$. But this violates the trichotomy known to hold for the ordering $<$ on the integers by Exercise \ref{exr:3.9a}, a contradiction. Second, suppose for the sake of contradiction that $p\lQ q$ and $p=q$. By the definition of $\lQ$, it follows from the first assumption that $ad<bc$. Additionally, by Exercise \ref{exr:2.6}, it follows from the second assumption that $(a,b)\sim(c,d)$, implying by Exercise \ref{exr:2.2e} that $ad=bc$. But once again, the simultaneous results that $ad<bc$ and $ad=bc$ violate the trichotomy of the integers, a contradiction. The proof of the third case is symmetric to that of the second.\par
            We now show that \emph{at least one} of the three statements is always true. Let $p,q$ be arbitrary elements of $\Q$, let $(a,b)\in p$, and let $(c,d)\in q$. Suppose for the sake of contradiction that $p\not\lQ q$, $q\not\lQ p$, and $p\neq q$. Since $p\not\lQ q$, we have that $ad\not<bc$. Similarly, since $q\not\lQ p$, we have $cb\not<da$. Equivalently, $bc\not<ad$. Lastly, since $p\neq q$, Exercise \ref{exr:2.6} implies that $(a,b)\nsim(c,d)$. It follows by Exercise \ref{exr:2.2e} that $ad\neq bc$. To recap, for the integers $ad$ and $bc$, we have $ad\not<bc$, $bc\not<ad$, and $ad\neq bc$. But by Exercise \ref{exr:3.9a}, $ad<bc$, $bc<ad$, or $ad=bc$, a contradiction.\par\smallskip
            To prove that $\lQ$ is transitive, it will suffice to show that for all $p,q,r\in\Q$, if $p\lQ q$ and $q\lQ r$, then $p\lQ r$. Let $p,q,r$ be arbitrary elements of $\Q$ for which it is true that $p\lQ q$ and $q\lQ r$, let $(a,b)\in p$ be such that $0<b$, let $(c,d)\in q$ be such that $0<d$, and let $(e,f)\in r$ such that $0<f$. By the definition of $\lQ$, we have $ad<bc$ and $cf<de$. Since $0<f$ and $0<b$, we can multiply both sides of the inequalities by $b$ or $f$ without affecting the truth of the statement (see Script \ref{sct:0}). Thus, we may create the inequalities $adf<bcf$ and $bcf<bde$. So $adf<bde$ by Definition \ref{dfn:3.1}, implying that $af<be$ by the cancellation law (which we may use since $0<d$). It follows by the definition of $\lQ$ that $p\lQ r$.
        \end{proof}
        \item \label{exr:3.9d}Show that $\Q$ with the ordering $\lQ$ satisfies Axioms 1-3.
        \begin{proof}
            Clearly, $\Q$ is a nonempty set, so Axiom \ref{axm:3.1} is immediately satisfied.\par
            By Exercise \ref{exr:3.9c}, $\Q$ has an ordering, so Axiom \ref{axm:3.2} is satisfied.\par
            Axiom \ref{axm:3.3} asserts that $\Q$ must have no first or last point. Suppose for the sake of contradiction that $\Q$ has some first point $p$. Then by Definition \ref{dfn:3.3}, $p\lQ x$ or $p=x$ for all $x\in\Q$. Let $(a,b)\in p$ be such that $0<b$ (see Exercise \ref{exr:3.9b}). Under the closure of integer subtraction, $(a-1)\in\Z$, so $\eqclass{a-1}{b}\in\Q$. Since $ba=ba-b+b=b(a-1)+b$ where $b\in\N$ since $b\in\Z$ and $0<b$, Definition \ref{dfn:3.8} implies that $(a-1)b<ba$. It follows by the definition of $\lQ$ that $\eqclass{a-1}{b}\lQ\eqclass{a}{b}=p$, a contradiction. The argument is symmetric for the last point.
        \end{proof}
    \end{enumerate}
\end{exercise}

\begin{definition}\label{dfn:3.10}
    If $a,b\in C$ and $a<b$, then the set of points between $a$ and $b$ is called a \textbf{region} and denoted by $\underline{ab}$.
\end{definition}

\begin{remark}\label{rmk:3.11}
    One often sees the notation $(a,b)$ for regions. We will reserve the notation $(a,b)$ for ordered pairs in a product $A\times B$. These are very different things.
\end{remark}

\begin{theorem}\label{trm:3.12}
    If $x$ is a point of a continuum $C$, then there exists a region $\underline{ab}$ such that $x\in\underline{ab}$.
    \begin{proof}
        Let $x$ be an arbitrary point in a continuum $C$. By Axiom \ref{axm:3.2}, $C$ has an ordering $<$, which we will frequently make use of throughout the remainder of this proof. By Axiom \ref{axm:3.3}, $C$ has no first or last points, so it cannot be true that $x\leq y$ for all $y\in C$, nor can it be true that $x\geq y$ for all $y\in C$. This implies that there exists an $a\in C$ such that $a<x$ and that there exists a $b\in C$ such that $b>x$. Since $a<x$ and $x<b$, Definition \ref{dfn:3.6} implies that $x$ is between $a$ and $b$. Note also that by Definition \ref{dfn:3.1} (specifically transitivity), $a<b$. Therefore, since $a,b\in C$, $a<b$, and $x$ is between $a$ and $b$, Definition \ref{dfn:3.10} implies that $x\in\underline{ab}$.
    \end{proof}
\end{theorem}

\begin{definition}\label{dfn:3.13}
    Let $A$ be a subset of a continuum $C$. A point $p$ of $C$ is called a \textbf{limit point} of $A$ if every region $R$ containing $p$ has nonempty intersection with $A\setminus\{p\}$. Explicitly, this means:
    \begin{center}
        for every region $R$ with $p\in R$, we have $R\cap(A\setminus\{p\})\neq\emptyset$.
    \end{center}
    Notice that we do not require that a limit point $p$ of $A$ be an element of $A$. We will use the notation $LP(A)$ to denote the set of limit points of $A$.
\end{definition}

\begin{theorem}\label{trm:3.14}
    If $p$ is a limit point of $A$ and $A\subset B$, then $p$ is a limit point of $B$.
    \begin{lemma*}
        Let $A,B,C$ be sets such that $A\subset B$. Then $A\cap C\subset B\cap C$.
        \begin{proof}
            Let $x$ be an arbitrary element of $A\cap C$. By Definition \ref{dfn:1.6}, this implies that $x\in A$ and $x\in C$. Since $x\in A$ and $A\subset B$, Definition \ref{dfn:1.3} implies that $x\in B$. Thus, $x\in B$ and $x\in C$, so $x\in B\cap C$ by Definition \ref{dfn:1.6}.
        \end{proof}
    \end{lemma*}
    \begin{proof}
        To prove that a limit point $p$ of $A\subset B$ is a limit point of $B$, Definition \ref{dfn:3.13} tells us that it will suffice to show that for every region $R$ with $p\in R$, we have $R\cap(B\setminus\{p\})\neq\emptyset$. Let $p$ be a limit point of $A$, and let $R$ be an arbitrary region with $p\in R$. Then by Definition \ref{dfn:3.13}, we have $R\cap(A\setminus\{p\})\neq\emptyset$. Thus, by Definition \ref{dfn:1.8}, there is an element $x\in R\cap(A\setminus\{p\})$. Since $A\setminus\{p\}\subset B\setminus\{p\}$ (because $A\subset B$ and $\{p\}=\{p\}$), it follows by the lemma that $R\cap(A\setminus\{p\})\subset R\cap(B\setminus\{p\})$. Consequently, by Definition \ref{dfn:1.3}, the previously referenced object $x\in R\cap(A\setminus\{p\})$ is also an element of $R\cap(B\setminus\{p\})$. Thus, by Definition \ref{dfn:1.8}, $R\cap(B\setminus\{p\})\neq\emptyset$, as desired.
    \end{proof}
\end{theorem}

\begin{definition}\label{dfn:3.15}\marginnote{10/27:}
    If $\underline{ab}$ is a region in a continuum $C$, then $C\setminus(\{a\}\cup\underline{ab}\cup\{b\})$ is called the \textbf{exterior} of $\underline{ab}$ and is denoted by $\ext\underline{ab}$.
\end{definition}

\begin{lemma}\label{lem:3.16}
    If $\underline{ab}$ is a region in a continuum $C$, then
    \begin{equation*}
        \ext\underline{ab} = \{x\in C\mid x<a\}\cup\{x\in C\mid b<x\}
    \end{equation*}
    \begin{proof}
        To prove that $\ext\underline{ab}=\{x\in C\mid x<a\}\cup\{x\in C\mid b<x\}$, Definition \ref{dfn:3.15} tells us that it will suffice to show that $C\setminus(\{a\}\cup\underline{ab}\cup\{b\})=\{x\in C\mid x<a\}\cup\{x\in C\mid b<x\}$. To do this, Definition \ref{dfn:1.2} tells us that we must verify that every element $y\in C\setminus(\{a\}\cup\underline{ab}\cup\{b\})$ is an element of $\{x\in C\mid x<a\}\cup\{x\in C\mid b<x\}$ and vice versa. Let's begin.\par
        First, let $y$ be an arbitrary element of $C\setminus(\{a\}\cup\underline{ab}\cup\{b\})$. By Definition \ref{dfn:1.11}, this implies that $y\in C$ and $y\notin\{a\}\cup\underline{ab}\cup\{b\}$. The latter result implies by Definition \ref{dfn:1.5} that $y\notin\{a\}$, $y\notin\underline{ab}$, and $y\notin\{b\}$. Since $y\notin\{a\}$ and $y\notin\{b\}$, we know that $y\neq a$ and $y\neq b$. Furthermore, since $y\notin\underline{ab}$, Definition \ref{dfn:3.10} asserts that $y$ is not between $a$ and $b$. Thus, by Definition \ref{dfn:3.6} and the fact that $a<b$ (i.e., case ii of Definition \ref{dfn:3.6} does not apply), we have that $y\leq a$ or $y\geq b$. But as previously established, $y\neq a$ and $y\neq b$, so it must be that $y<a$ or $y>b$. We divide into two cases. If $y<a$, then this fact combined with the fact that $y\in C$ implies that $y\in\{x\in C\mid x<a\}$. Therefore, by Definition \ref{dfn:1.5}, $y\in\{x\in C\mid x<a\}\cup\{x\in C\mid b<x\}$, as desired. Similarly, if $y>b$, we have $y\in\{x\in C\mid b<x\}$, meaning that $y\in\{x\in C\mid x<a\}\cup\{x\in C\mid b<x\}$, as desired.\par
        Now let $y$ be an arbitrary element of $\{x\in C\mid x<a\}\cup\{x\in C\mid b<x\}$. By Definition \ref{dfn:1.5}, this implies that $y\in\{x\in C\mid x<a\}$ or $y\in\{x\in C\mid b<x\}$. We divide into two cases. Suppose first that $y\in\{x\in C\mid x<a\}$. Then $y\in C$ and $y<a$. Since $y<a$, Definition \ref{dfn:3.1} implies that $y\neq a$, i.e., $y\notin\{a\}$. Since $y<a$ and $a<b$, Definition \ref{dfn:3.1} implies that $y<b$. Thus, for similar reasons to before, $y\neq b$, i.e., $y\notin\{b\}$. Lastly, since $a<b$, for $y$ to be between $a$ and $b$, Definition \ref{dfn:3.6} implies that we must have $a<y$ and $y<b$. But $y<a$, so it must be that $y$ is not between $a$ and $b$. Thus, by Definition \ref{dfn:3.10}, $y\notin\underline{ab}$. Since $y\notin\{a\}$, $y\notin\underline{ab}$, and $y\notin\{b\}$, Definition \ref{dfn:1.5} asserts that $y\notin\{a\}\cup\underline{ab}\cup\{b\}$. Therefore, since we also have $y\in C$ as previously established, Definition \ref{dfn:1.11} implies that $y\in C\setminus(\{a\}\cup\underline{ab}\cup\{b\})$, as desired. The proof is symmetric if $y\in\{x\in C\mid b<x\}$.
    \end{proof}
\end{lemma}

\begin{lemma}\label{lem:3.17}
    No point in the exterior of a region is a limit point of that region. No point of a region is a limit point of the exterior of that region.
    \begin{proof}
        We will take this one claim at a time, starting with the first listed claim.\par
        Let $\underline{ab}$ be an arbitrary region of a continuum $C$. To prove that no point in the exterior of $\underline{ab}$ is a limit point of $\underline{ab}$, Definition \ref{dfn:3.13} tells us that it will suffice to show that for all points $p\in\ext\underline{ab}$, there exists some region $R$ with $p\in R$ such that $R\cap(\underline{ab}\setminus\{p\})=\emptyset$. Let $p$ be an arbitrary element of $\ext\underline{ab}$. Then by Lemma \ref{lem:3.16} and Definition \ref{dfn:1.5}, $p\in\{x\in C\mid x<a\}$ or $p\in\{x\in C\mid b<x\}$. We divide into two cases. Suppose first that $p\in\{x\in C\mid x<a\}$. It follows that $p<a$, so let $c\in C$ be a point such that $c<p$ (Axiom \ref{axm:3.3} and Definition \ref{dfn:3.3} imply that such a point exists). Since $c<p$ and $p<a$, Definition \ref{dfn:3.6} implies that $p$ is between $c$ and $a$. Thus, Definition \ref{dfn:3.10} implies that $p\in\underline{ca}$. Now suppose for the sake of contradiction that for some object $x$, $x\in\underline{ca}\cap(\underline{ab}\setminus\{p\})$. By Definition \ref{dfn:1.6}, this implies that $x\in\underline{ca}$ and $x\in\underline{ab}\setminus\{p\}$. Since $x\in\underline{ca}$, Definitions \ref{dfn:3.10} and \ref{dfn:3.6} imply that $c<x$ and $x<a$. Additionally, since $x\in\underline{ab}\setminus\{p\}$, Definition \ref{dfn:1.11} implies that $x\in\underline{ab}$ and $x\notin\{p\}$, so with respect to the former claim, $a<x$ and $x<b$, as before. But by Definition \ref{dfn:3.1}, we cannot have $x<a$ and $a<x$, so we have arrived at a contradiction. Therefore, $x\notin\underline{ca}\cap(\underline{ab}\setminus\{p\})$ for any $x$, proving by Definition \ref{dfn:1.8} that $\underline{ca}\cap(\underline{ab}\setminus\{p\})=\emptyset$, as desired. The proof is symmetric if $p\in\{x\in C\mid b<x\}$.\par
        Let $\underline{ab}$ be an arbitrary region of a continuum $C$. To prove that no point of $\underline{ab}$ is a limit point of the exterior of $\underline{ab}$, Definition \ref{dfn:3.13} tells us that it will suffice to show that for all points $p\in\underline{ab}$, there exists some region $R$ with $p\in R$ such that $R\cap(\ext\underline{ab}\setminus\{p\})=\emptyset$. Let $p$ be an arbitrary element of $\underline{ab}$. Then $\underline{ab}$ is actually a $p$-containing region having empty intersection with $\ext\underline{ab}\setminus\{p\}$, as will now be proven. Suppose for the sake of contradiction that for some object $x$, $x\in\underline{ab}\cap(\ext\underline{ab}\setminus\{p\})$. By Definition \ref{dfn:3.15}, this implies that $x\in\underline{ab}\cap((C\setminus(\{a\}\cup\underline{ab}\cup\{b\}))\setminus\{p\})$. Thus, by Definition \ref{dfn:1.6}, $x\in\underline{ab}$ and $x\in(C\setminus(\{a\}\cup\underline{ab}\cup\{b\}))\setminus\{p\}$. Consequently, by consecutive applications of Definition \ref{dfn:1.11}, $x\in C$, $x\notin\{a\}\cup\underline{ab}\cup\{b\}$, and $x\notin\{p\}$. With respect to the middle of the three previous results, Definition \ref{dfn:1.5} implies that $x\notin\{a\}$, $x\notin\underline{ab}$, and $x\notin\{b\}$. But we have previously demonstrated that $x\in\underline{ab}$, a contradiction. Therefore, $x\notin\underline{ab}\cap(\ext\underline{ab}\setminus\{p\})$ for any $x$, proving by Definition \ref{dfn:1.8} that $\underline{ab}\cap(\ext\underline{ab}\setminus\{p\})=\emptyset$, as desired.
    \end{proof}
\end{lemma}

\begin{theorem}\label{trm:3.18}
    If two regions have a point $x$ in common, their intersection is a region containing $x$.
    \begin{proof}
        Let $\underline{ab}$ and $\underline{cd}$ be two regions of a continuum $C$ such that for some point $x\in C$, $x\in\underline{ab}$ and $x\in\underline{cd}$. We divide into two cases ($a\leq c$ and $b\leq d$, and $a\leq c$ and $b>d$) WLOG\footnote{The other two cases ($a>c$ and $b\leq d$, and $a>c$ and $b>d$) can be encapsulated in the first two by switching the names of the two regions.}.\par
        Suppose first that $a\leq c$ and $b\leq d$. We seek to prove that $\underline{ab}\cap\underline{cd}=\underline{cb}$ where $\underline{cb}$ is clearly a region, and that $x\in\underline{cb}$. To prove that $\underline{ab}\cap\underline{cd}=\underline{cb}$, Definition \ref{dfn:1.2} tells us that it will suffice to show that every element $y\in\underline{ab}\cap\underline{cd}$ is an element of $\underline{cb}$ and vice versa. Suppose first that $y$ is an arbitrary element of $\underline{ab}\cap\underline{cd}$. Then by Definition \ref{dfn:1.6}, $y\in\underline{ab}$ and $y\in\underline{cd}$. Thus, by Definitions \ref{dfn:3.10} and \ref{dfn:3.6}, $a<y$, $y<b$, $c<y$, and $y<d$. Since $c<y$ and $y<b$, it follows by Definitions \ref{dfn:3.6} and \ref{dfn:3.10} that $y\in\underline{cb}$, as desired. Now suppose that $y\in\underline{cb}$. Then by Definitions \ref{dfn:3.10} and \ref{dfn:3.6}, $c<y$ and $y<b$. Since $a\leq c$ (by assumption) and $c<y$, $a<y$ (if $a=c$, then $c<y$ implies $a<y$ by substitution; if $a<c$ and $c<y$, then Definition \ref{dfn:3.1} implies $a<y$\footnote{This justification and simple variations thereof, although used again, will not be stated again.}). Thus, since $a<y$ and $y<b$, Definitions \ref{dfn:3.6} and \ref{dfn:3.10} imply that $y\in\underline{ab}$. Similarly, $y<b$ and $b\leq d$ (by assumption) together imply that $y<d$. Thus, since $c<y$ and $y<d$, Definitions \ref{dfn:3.6} and \ref{dfn:3.10} imply that $y\in\underline{cd}$. Since $y\in\underline{ab}$ and $y\in\underline{cd}$, Definition \ref{dfn:1.6} implies that $y\in\underline{ab}\cap\underline{cd}$, as desired. Lastly, since $x\in\underline{ab}$ and $x\in\underline{cd}$ by hypothesis, Definition \ref{dfn:1.6} implies that $x\in\underline{ab}\cap\underline{cd}$, which means by Definition \ref{dfn:1.2} and the above that $x\in\underline{cb}$, as desired.\par
        Now suppose that $a\leq c$ and $b>d$. We seek to prove that $\underline{ab}\cap\underline{cd}=\underline{cd}$ where $\underline{cd}$ is clearly a region, and that $x\in\underline{cd}$. To prove that $\underline{ab}\cap\underline{cd}=\underline{cd}$, Theorem \ref{trm:1.7a} tells us that it will suffice to show that $\underline{ab}\cap\underline{cd}\subset\underline{cd}$ and $\underline{cd}\subset\underline{ab}\cap\underline{cd}$. But by Theorem \ref{trm:1.7c}, $\underline{ab}\cap\underline{cd}\subset\underline{cd}$, so all that's left to prove is that $\underline{cd}\subset\underline{ab}\cap\underline{cd}$. Definition \ref{dfn:1.3} tells us that we may do this by demonstrating that every element $y\in\underline{cd}$ is an element of $\underline{ab}\cap\underline{cd}$. Let $y$ be an arbitrary element of $\underline{cd}$. Then by Definitions \ref{dfn:3.10} and \ref{dfn:3.6}, $c<y$ and $y<d$. Since $a\leq c$ (by assumption) and $c<y$, $a<y$, and since $y<d$ and $d<b$, $y<b$. Thus, since $a<y$ and $y<b$, Definitions \ref{dfn:3.6} and \ref{dfn:3.10} imply that $y\in\underline{ab}$. This fact combined with the hypothesis that $y\in\underline{cd}$ implies by Definition \ref{dfn:1.6} that $y\in\underline{ab}\cap\underline{cd}$, as desired. Lastly, since $\underline{cd}$ \emph{is} the intersection of $\underline{ab}$ and $\underline{cd}$, and $x\in\underline{cd}$ by hypothesis, $x$ is clearly an element of the intersection of $\underline{ab}$ and $\underline{cd}$.
    \end{proof}
\end{theorem}

\begin{corollary}\label{cly:3.19}
    If $n$ regions $R_1,\dots,R_n$ have a point $x$ in common, then their intersection $R_1\cap\cdots\cap R_n$ is a region containing $x$.
    \begin{proof}
        We induct on $n$ from the base case $n_0=2$ using the form of induction described in Additional Exercise \ref{axr:0.2a}. For the base case $n=2$, let $R_1$ and $R_2$ be two regions that have a point $x$ in common. By Theorem \ref{trm:3.18}, it follows that their intersection $R_1\cap R_2$ is a region containing $x$, proving the base case. Now suppose inductively that we have proven the claim for some $n$, i.e., we know that if $n$ regions $R_1,\dots,R_n$ have a point $x$ in common, then their intersection $\bigcap_{k=1}^nR_k$ is a region containing $x$. We wish to prove that if $n+1$ regions $R_1,\dots,R_{n+1}$ have a point $x$ in common, then their intersection $\bigcap_{k=1}^{n+1}R_k$ is a region containing $x$. Let $R_1,\dots,R_{n+1}$ be $n+1$ regions that have a point $x$ in common. By the induction hypothesis, $\bigcap_{k=1}^nR_k$ is a region containing $x$. Since $\bigcap_{k=1}^nR_k$ and $R_{n+1}$ are both regions with a point $x$ in common, Theorem \ref{trm:3.18} applies and implies that $\left( \bigcap_{k=1}^nR_k \right)\cap R_{n+1}=\bigcap_{k=1}^{n+1}R_k$ is a region containing $x$, thus closing the induction.
    \end{proof}
\end{corollary}

\begin{theorem}\label{trm:3.20}
    Let $A,B$ be subsets of a continuum $C$. Then $p$ is a limit point of $A\cup B$ if and only if $p$ is a limit point of at least one of $A$ or $B$.
    \begin{proof}
        To prove that $p$ is a limit point of $A\cup B$ if and only if $p$ is a limit point of at least one of $A$ or $B$, we must prove the dual implications "$p\in LP(A)$ or $p\in LP(B)$ implies $p\in LP(A\cup B)$" and "$p\in LP(A\cup B)$ implies $p\in LP(A)$ or $p\in LP(B)$." The first implication will be proved directly, but the second one will be proved by contrapositive. Let's begin.\par
        Suppose first that $p$ is a limit point of $A$ or $B$. We divide into two cases. If $p\in LP(A)$, then since $A\subset A\cup B$ by Theorem \ref{trm:1.7}, Theorem \ref{trm:3.14} applies and implies that $p\in LP(A\cup B)$. The proof is symmetric if $p\in LP(B)$.\par
        Now suppose that $p\notin LP(A)$ and $p\notin LP(B)$. Then by the contrapositive of Definition \ref{dfn:3.13}, there exist regions $R_1$ and $R_2$ with $p\in R_1$ and $p\in R_2$ such that $R_1\cap(A\setminus\{p\})=\emptyset$ and $R_2\cap(B\setminus\{p\})=\emptyset$. Since $R_1$ and $R_2$ are two regions that have a point (namely $p$) in common, Theorem \ref{trm:3.18} asserts that $R_1\cap R_2$ is a region $R$ with $p\in R$. Additionally, since $R_1\cap R_2\subset R_1$ and $R_1\cap R_2\subset R_2$ by Theorem \ref{trm:1.7}, $R\subset R_1$ and $R\subset R_2$. Thus, $R\cap(A\setminus\{p\})=\emptyset$ and $R\cap(B\setminus\{p\})=\emptyset$. Now since $U\cup\emptyset=U$ for any set $U$, we know that
        \begin{align*}
            \emptyset &= R\cap(A\setminus\{p\})\\
            &= (R\cap(A\setminus\{p\}))\cup\emptyset
            \intertext{Substitute $\emptyset=R\cap(B\setminus\{p\})$.}
            &= (R\cap(A\setminus\{p\}))\cup(R\cap(B\setminus\{p\}))
            \intertext{Apply the fact that $X\cap(Y\cup Z)=(X\cap Y)\cup(X\cap Z)$ for any sets $X,Y,Z$.}
            &= R\cap((A\setminus\{p\})\cup(B\setminus\{p\}))
            \intertext{Apply the fact that $(X\setminus Z)\cup(Y\setminus Z)=(X\cup Y)\setminus Z$ for any sets $X,Y,Z$.}
            &= R\cap((A\cup B)\setminus\{p\})
        \end{align*}
        Since there exists a region $R$ with $p\in R$ such that $R\cap((A\cup B)\setminus\{p\})=\emptyset$, the contrapositive of Definition \ref{dfn:3.13} implies that $p\notin LP(A\cup B)$.
    \end{proof}
\end{theorem}

\begin{corollary}\label{cly:3.21}
    Let $A_1,\dots,A_n$ be $n$ subsets of a continuum $C$. Then $p$ is a limit point of $A_1\cup\cdots\cup A_n$ if and only if $p$ is a limit point of at least one of the sets $A_k$.
    \begin{proof}
        We induct on $n$ from the base case $n_0=2$ using the form of induction described in Additional Exercise \ref{axr:0.2a}. For the base case $n=2$, let $A_1$ and $A_2$ be two subsets of a continuum $C$. By Theorem \ref{trm:3.20}, it follows that $p$ is a limit point of $A_1\cup A_2$ if and only if $p$ is a limit point of at least one of $A_1$ or $A_2$, proving the base case. Now suppose inductively that we have proven the claim for some $n$, i.e., we know that if there exist $n$ subsets $A_1,\dots,A_n$ of a continuum $C$, then $p$ is a limit point of $\bigcup_{k=1}^nA_k$ if and only if $p$ is a limit point of at least one of the sets $A_k$. We wish to prove that if there exist $n+1$ subsets $A_1,\dots,A_{n+1}$ of a continuum $C$, then $p$ is a limit point of $\bigcup_{k=1}^{n+1}A_k$ if and only if $p$ is a limit point of at least one of the sets $A_k$. Let $A_1,\dots,A_n$ be $n+1$ subsets of a continuum $C$. By the induction hypothesis, $p$ is a limit point of $\bigcup_{k=1}^nA_k$ if and only if $p$ is a limit point of at least one of the sets $A_k$. Since $\bigcup_{k=1}^nA_k$ and $A_{n+1}$ are both subsets of a continuum $C$, Theorem \ref{trm:3.20} applies and implies that $p$ is a limit point of $\left( \bigcup_{k=1}^nA_k \right)\cup A_{n+1}=\bigcup_{k=1}^{n+1}A_k$ if and only if $p$ is a limit point of at least one of $\bigcup_{k=1}^nA_k$ or $A_{n+1}$. But the last two statements combined imply that $p$ is a limit point of $\bigcup_{k=1}^{n+1}A_k$ if and only if $p$ is a limit point of at least one of the sets $A_k$ where $1\leq k\leq n$ or $A_{n+1}$, i.e., $p$ is a limit point of $\bigcup_{k=1}^{n+1}A_k$ if and only if $p$ is a limit point of at least one of the sets $A_k$, thus closing the induction.
    \end{proof}
\end{corollary}

\begin{theorem}\label{trm:3.22}
    If $p$ and $q$ are distinct points of a continuum $C$, then there exist disjoint regions $R$ and $S$ containing $p$ and $q$, respectively.
    \begin{proof}
        WLOG, let $p<q$. Additionally, let $a<p$ and $b>q$ be points of $C$ (Axiom \ref{axm:3.3} and Definition \ref{dfn:3.3} imply that such points exist). We divide into two cases (no point $x\in C$ exists between $p$ and $q$, and there exists a point $x\in C$ between $p$ and $q$). Let's begin.\par
        Suppose first that no point $x\in C$ exists between $p$ and $q$. Let $R=\underline{aq}$ and let $S=\underline{pb}$. Thus, Definitions \ref{dfn:3.6} and \ref{dfn:3.10} imply by the facts that $a<p$ by definition and $p<q$ by hypothesis, and $p<q$ by hypothesis and $q<b$ by definition that $p\in R$ and $q\in S$, respectively. To prove that $R$ and $S$ are disjoint, Definition \ref{dfn:1.9} tells us that it will suffice to show that $R\cap S=\emptyset$. Suppose for the sake of contradiction that there exists an object $x\in R\cap S$. By Definition \ref{dfn:1.6}, this implies that $x\in R$ and $x\in S$ (and, hence, that $x\in C$). Thus, by consecutive applications of Definitions \ref{dfn:3.10} and \ref{dfn:3.6}, we have that $a<x$, $x<q$, $p<x$, and $x<b$. Since $p<x$ and $x<q$, Definition \ref{dfn:3.6} implies that $x$ is between $p$ and $q$. But by hypothesis, no point $x\in C$ exists between $p$ and $q$, a contradiction. Therefore, since $x\notin R\cap S$ for all $x$, Definition \ref{dfn:1.8} implies that $R\cap S=\emptyset$, as desired.\par
        Now suppose that there exists a point $x\in C$ between between $p$ and $q$. Let $R=\underline{ax}$ and let $S=\underline{xb}$. It follows from the hypothesis by Definition \ref{dfn:3.6} that $p<x$ and $x<q$. Thus, Definitions \ref{dfn:3.6} and \ref{dfn:3.10} imply by the facts that $a<p$ and $p<x$, and $x<q$ and $q<b$ that $p\in R$ and $q\in S$, respectively. To prove that $R$ and $S$ are disjoint, Definition \ref{dfn:1.9} tells us tht it will suffice to show that $R\cap S=\emptyset$. Suppose for the sake of contradiction that there exists an object $y\in R\cap S$. By Definition \ref{dfn:1.6}, this implies that $y\in R$ and $y\in S$ (and, hence, that $y\in C$). Thus, by consecutive applications of Definitions \ref{dfn:3.10} and \ref{dfn:3.6}, we have that $a<y$, $y<x$, $x<y$, and $y<b$. But by Definition \ref{dfn:3.1}, we cannot have $y<x$ and $x<y$, a contradiction. Therefore, since $x\notin R\cap S$ for all $x$, Definition \ref{dfn:1.8} implies that $R\cap S=\emptyset$, as desired.
    \end{proof}
\end{theorem}

\begin{corollary}\label{cly:3.23}\marginnote{\emph{10/29:}}
    A subset of a continuum $C$ consisting of one point has no limit points.
    \begin{proof}
        Let $\{x\}\subset C$. To prove that $\{x\}$ has no limit points, Definition \ref{dfn:3.13} tells us that it will suffice to show that for all $p\in C$, there exists a region $R$ with $p\in R$ such that $R\cap(\{x\}\setminus\{p\})=\emptyset$. Let $p$ be an arbitrary point in $C$. We divide into two cases ($p=x$ and $p\neq x$; Definition \ref{dfn:3.1} guarantees that these two cases account for all $p\in C$). Suppose first that $p=x$. If we let $R$ be any region of $C$, it follows that
        \begin{align*}
            R\cap(\{x\}\setminus\{p\}) &= R\cap(\{x\}\setminus\{x\})\\
            &= R\cap\emptyset\\
            &= \emptyset
        \end{align*}
        as desired. Now suppose that $p\neq x$. Since $p$ and $x$ are distinct points, Theorem \ref{trm:3.22} applies and implies that there exist disjoint regions $R$ and $S$ containing $p$ and $x$, respectively. Consequently, by Definition \ref{dfn:1.9}, $x\notin R$. Thus,
        \begin{align*}
            R\cap(\{x\}\setminus\{p\}) &= R\cap\{x\}\\
            &= \emptyset
        \end{align*}
        as desired.
    \end{proof}
\end{corollary}

\begin{theorem}\label{trm:3.24}
    A finite subset $A$ of a continuum $C$ has no limit points.
    \begin{proof}
        We divide into two cases ($A=\emptyset$ and $A\neq\emptyset$).\par
        Suppose that $A=\emptyset$. Then for an arbitrary $p\in C$ and any region $R$ of $C$,
        \begin{align*}
            R\cap(A\setminus\{p\}) &= R\cap(\emptyset\setminus\{p\})\\
            &= R\cap\emptyset\\
            &= \emptyset
        \end{align*}
        proving by the contrapositive of Definition \ref{dfn:3.13} that $p$ is not a limit point of $A$, as desired.\par
        Now suppose that $A\neq\emptyset$. Since $A$ is finite, i.e., is a set of $n$ distinct points of a continuum $C$ for some $n\in\N$, Theorem \ref{trm:3.5} applies and implies that the symbols $a_1,\dots,a_n$ can be assigned to each point of $A$. It follows that we can write $A=\bigcup_{k=1}^n\{a_k\}$. Now suppose for the sake of contradiction that for some $p\in C$, $p\in LP(A)$. Then by Corollary \ref{cly:3.21}, $p$ is a limit point of at least one of the sets $\{a_k\}$. But this contradicts Corollary \ref{cly:3.23}, which asserts that no singleton set has limit points. Therefore, no point $p$ of $C$ is a limit point of $A$, i.e., $A$ has no limit points, as desired.
    \end{proof}
\end{theorem}

\begin{corollary}\label{cly:3.25}
    If $A$ is a finite subset of a continuum $C$ and $x\in A$, then there exists a region $R$ containing $x$, such that $A\cap R=\{x\}$.
    \begin{proof}
        Since $A$ is a finite subset of a continuum $C$, Theorem \ref{trm:3.24} implies that $A$ has no limit points. Thus, for an arbitrary $x\in A$, $x$ is not a limit point of $A$. Consequently, Definition \ref{dfn:3.13} implies that there exists a region $R$ of $C$ with $x\in R$ such that $R\cap(A\setminus\{x\})=\emptyset$. Moreover, since $x\in A$ and $x\in R$, Definition \ref{dfn:1.6} implies that $x\in A\cap R$. Now suppose for the sake of contradiction that there exists some object $y\in A\cap R$ such that $y\neq x$. Then by Definition \ref{dfn:1.6}, $y\in A$ and $y\in R$. The former discovery combined with the fact that $y\neq x$, i.e., $y\notin\{x\}$ reveals that $y\in A\setminus\{x\}$ by Definition \ref{dfn:1.11}. Thus, since $y\in R$ and $y\in A\setminus\{x\}$, $y\in R\cap(A\setminus\{x\})$. It follows by Definition \ref{dfn:1.2} that since $R\cap(A\setminus\{x\})=\emptyset$ that $y\in\emptyset$, but this contradicts Definition \ref{dfn:1.8}. Therefore, $x$ is the only element of $A\cap R$, so $A\cap R=\{x\}$, as desired.
    \end{proof}
\end{corollary}

\begin{theorem}\label{trm:3.26}
    If $p$ is a limit point of $A$ and $R$ is a region containing $p$, then the set $R\cap A$ is infinite.
    \begin{proof}
        Suppose for the sake of contradiction that $R\cap A$ is finite. Then by Theorem \ref{trm:3.24}, $R\cap A$ has no limit points. Notably, this implies that $p$ is not a limit point of $R\cap A$. It follows by Definition \ref{dfn:3.13} that there exists some region $S$ with $p\in S$ such that
        \begin{align*}
            \emptyset &= S\cap((R\cap A)\setminus\{p\})
            \intertext{Since $p\in S$ and $p\in R$, Theorem \ref{trm:3.18} implies that $S\cap R$ is a region containing $x$. But since the above}
            &= (S\cap R)\cap(A\setminus\{p\})
        \end{align*}
        where $S\cap R$ is a $p$-containing region, Definition \ref{dfn:3.13} implies that $p$ is not a limit point of $A$. But this contradicts the hypothesis that $p$ is a limit point of $A$. Thus, $R\cap A$ must be infinite.
    \end{proof}
\end{theorem}

\begin{exercise}\label{exr:3.27}
    Find realizations of a continuum $(C,<)$. That is, find concrete sets $C$ endowed with a relation $<$ satisfying all of the axioms (so far). Are they the same? What does "the same" mean here?
    \begin{proof}[Explanation]
        $\Z$ and $\Q$ are complete realizations of a continuum $(C,<)$ (see Exercise \ref{exr:3.9}). As to the other part of the question, it is hard to define what "the same" means with respect to infinite sets. While there exists a bijection between them (there exist bijections $f:\Q\to\N$ and $g:\N\to\Z$ by Theorem \ref{trm:2.11} and Exercise \ref{exr:1.36}, respectively, so $g\circ f:\Q\to\Z$ is bijective by Proposition \ref{prp:1.26}), I would argue that they are \emph{not} the same since when ordered in the common sense, there are elements of $\Q$ between each element of $\Z$. Additionally, the orderings defined on the two sets are not the same. Lastly, $\Q$ is dense\footnote{For any two elements $p,q\in Q$, there exists an element $r$ between $p$ and $q$.} while $\Z$ is not.
    \end{proof}
\end{exercise}




\end{document}